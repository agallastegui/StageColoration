\documentclass[10pt,a4paper]{article}
\usepackage[utf8]{inputenc}
\usepackage[french]{babel}
\usepackage[T1]{fontenc}
\usepackage{amsmath}
\usepackage{amsfonts}
\usepackage{amssymb}
\usepackage{makeidx}
\usepackage{graphicx}
\usepackage{color}
\usepackage[left=2cm,right=2cm,top=2cm,bottom=2cm]{geometry}
\newtheorem{definition}{Définition}
\newtheorem{theorem}{Théorème}
\newtheorem{proposition}{Proposition}
\newtheorem{exemple}{Exemple}
\newtheorem{lemme}{Lemme}
\newtheorem{conjecture}{Conjecture}
\newtheorem{corollaire}{Corollaire}


\newcommand{\ep}{{\hfill $\square$}}

\title{Coloration des Arêtes des Graphes Planaires}
\author{Antonio Gallastegui
 }
 \date{
 Mémoire de Master 2 MIMSE \\
  Modélisation, Ingénierie Mathématique, Statistique et Économique \\
  spécialité Recherche Opérationnelle et Aide à la Décision \\\bigskip
  stage de recherche au LaBRI\\
  Laboratoire Bordelais de Recherche en Informatique \\
  responsable de stage : Franti\v sek Kardo\v s  \\
  thème : Graphes et optimisation \\
  équipe : Combinatoire et algorithmique  \\
 }
\addtolength{\parskip}{5pt}

\begin{document}
\thispagestyle{empty}
\maketitle

\begin{abstract}
Dans la démonstration du fameux Théorème des quatre couleurs, la question principale est traduite en une recherche d'une 3-coloration des arêtes d'un graphes planaire cubique. L'existence d'une telle coloration est démontré en utilisant la méthode de déchargement ; la réductibilité de certaines configurations est démontrée à l'aide d'un ordinateur. En s'inspirant de cette technique, pour attaquer une question ouverte sur la coloration des arêtes des graphes planaires (4-arête-colorabilité des graphes planaires de degré max 4 sans triangles), nous avons conçu et implémenté un programme qui permettra de vérifier la réductibilité de configurations d'une façon automatisée. 
\end{abstract}

\section*{Introduction}
La théorie des graphes est un domaine des mathématiques discrètes qui remonte au XVIIIème siècle et le célèbre problème des ponts de Könisberg (Kaliningrad). Ce problème consiste à essayer de construire un promenade qui commence et finit au même point, tout en parcourant les 7 ponts qu'il y avait dans la ville à l'époque, chacun une seule fois. C'est un domaine très étendu et appliqué parmi des autres sciences. 

Un des problèmes principaux au sein de la théorie des graphes est le problème de coloration des graphes. Le champ d'applications de la coloration des graphes couvre notamment le problème de l'attribution de fréquences dans les télécommunications, la conception de puces électroniques ou l'allocation de registres en compilation.

Le problème étudié est un problème de coloration des graphes. Plus particulièrement, il s'agit d'une conjecture disant qu'il est toujours possible  de colorier avec quatre couleurs les arêtes des graphes d'une certaine classe. Pour aborder ce problème on a pensé à utiliser les méthodes utilisées pour résoudre un autre problème de coloration des graphes très connu, le Théorème de quatre couleurs. 

Lé mémoire est structuré de la façon suivante :
Dans le chapitre \ref{chap:pres}, nous présentons les définitions, les théorèmes et les exemples pour bien se placer dans le domaine. Nous illustrons sur deux exemples d'une complexité non-triviale les deux méthodes principales utilisées aussi dans la preuve du Théorème de quatre couleurs : la réductibilité de configurations et la méthode de déchargement. Le concept de réductibilité est décrit plus détaillé, par ce que c'est pour avancer dans cette démarche qu'on a développer un programme.

Le chapitre \ref{chap:easy} contient ...

Le chapitre \ref{chap:meth} est dédié à la description du programme permettant de vérifier la réductibilité de configurations. ... 


etc. 
à compléter une fois le mémoire tout rempli.
%Une fois tous les aspects de base de la théorie de graphes soient bien expliqués, il faudra mentionner aussi le théorème de 4 couleurs, un ancien problème très connus qui se soutient en la réductibilité aussi et duquel on s'est inspiré pour aborder notre problème. On va analyser la façon de traiter les graphes et la technique de réduction de ce problème pour essayer de trouver une moyenne de réduire nos graphes.

%Ensuite on passe donc, à décrire les aspects techniques mentionnés ainsi que les différentes méthodes qui seront utilisées.

\section{Présentation de Sujet}
\label{chap:pres}

Dans cette partie nous faisons, d'abord, un petit rappel des fondamentaux de la théorie des graphes ainsi que des explications plus profondes des méthodes et techniques nécessaires pour bien comprendre le sujet du problème traité. 
Ensuite, la question principale étudiée est introduite avec son contexte.
Nous allons, quand même, commenter la démonstration du Théorème de quatre couleurs, l'expliquer et évoquer des ressemblances avec notre problème. 
%Cette partie sera donc, consacrée à la mise en place des compétences et à la explication des conditions de sujet.

\subsection{Notions générales de la théorie des graphes}

Dans cette partie nous rappelons quelques notions de la théorie de graphes,  des définitions de base et des concepts qui seront nécessaires pour la suite. 

\begin{definition}
Un \emph{graphe} $G$ est un couple $G = (V,E)$ d'ensembles finis, où 
\begin{itemize}
\item $V=V(G)$ est l'ensemble de \emph{sommets (Vertices)} de $G$, et
\item $E=E(G)$ est un ensemble de pairs de sommets, appelées des \emph{arêtes (Edges)} de $G$.
\end{itemize}
Si $e = uv \in E(G)$, on dit que $u$ et $v$ sont \emph{adjacent} (ou \emph{voisins}), d'ailleurs, l'arête $e$ est \emph{incidente} aux sommets $u$ et $v$.

Deux arêtes $e,e'\in E(G)$ sont \emph{adjacentes} si $e=uv$ et $e'=uv'$ pour $u,v,v'\in V(G)$, $v\ne v'$.
\end{definition}
  

\begin{definition}
Le \emph{degré} $d_{G}(v)$ d'un sommet $v$ d'un graphe $G$ est le nombre de voisins de $v$. Le degré maximum d'un graphe $G$ (le maximum des degrés de ses sommets) est noté $\Delta(G)$ et le degré min $\delta(G)$.
\end{definition}

\begin{definition}
Soit $G=(V(G),E(G))$ un graphe simple. Alors un \emph{couplage} $M$ est un ensemble d'arêtes deux à deux non adjacentes. C-à-d que $M$ est une partie de l'ensemble $E$ des arêtes telle que pour tout $(e,e') \in M^2$, $e \neq e' \Rightarrow e \cap e' = \varnothing$.

Un \emph{couplage maximum} est un couplage contenant le plus grande nombre possible d'arêtes.

Un \emph{couplage maximal} est un couplage du graphe tel que toute arête du graphe possède au moins un des sommets incidentes à elle qui est au même temps incidente à une arête du couplage. c-à-d $M$ est un couplage auquel ajoutant n'importe quelle arête $e \in E(G)$, $M \cup e$ n'est plus un couplage.

Un \emph{couplage parfait} est un couplage du graphe tel que tout sommet du graphe est incident à une seule arête du couplage.
\end{definition}

\begin{definition}
Soit $G$ un graphe non orienté. Un \emph{cycle} est une suite ou une chaîne des arêtes du graphe dont le premier sommet incident à la première arête et le dernier sommet incident à la dernière arête sont le même.
\end{definition}

\begin{definition}
Soit $G=(V,E)$ un graphe. On dit que $G$ est un graphe \emph{planaire} si et seulement s'il peut être dessiné sur le plan sans croisement des arêtes. 

Étant donné un plongement d'un graphe planaire dans le plan, une \emph{face} est une région connexe délimitée par des arêtes. Il y a toujours une face infinie, appelée ainsi la \emph{face extérieure} de $G$.

On note $F(G)$ l'ensemble des faces de $G$. La \emph{taille} d'une face $f$, $d_G(f)$, est la longueur de la chaîne fermée (pas nécessairement élémentaire) parcourant la frontière de la face $f$.
\end{definition}

\begin{theorem}[Euler, 1750]
Soit $G$ un graphe planaire connexe qui a $n$ sommets, $m$ arêtes et $f$ faces. Alors
$$
n - m + f = 2.
$$
\end{theorem}

\begin{figure}[ht]
\centerline{
\includegraphics[scale=1]{./Figures/Fig-def.1}
\hfil
\includegraphics[scale=1]{./Figures/Fig-def.2}
}
\caption{Un exemple d'un graphe de degré minimum 2 et degré maximum 4 (gauche). Le même graphe plongé dans le plan sans croisements d'arêtes (droite). Il a quatre faces triangulaires et une face extérieure de taille 8.}
\label{fig:ex}
\end{figure}

Par exemple, le graphe de Figure \ref{fig:ex} il a $n=7$ sommets, $m=10$ arêtes et $f=5$ faces. En effet, $7 - 10 + 5 = 2$.

Il est facile de déduire de la formule d'Euler l'énoncé suivant.

\begin{proposition}
Tout graphe planaire contient un sommet de degré au plus 5.
\end{proposition}


Si un graphe planaire a une face de taille supérieur à 3, on peut insérer une arête (une diagonale de la face) afin de la diviser en deux faces de moindre taille. Un graphe planaire obtenu en appliquant cette opération tant que possible n'a que des faces triangulaires, il est appelé ainsi une \emph{triangulation}. Étant donné un graphe planaire $G$, quelque soit la façon d'ajouter des arêtes, toute triangulation obtenu à partir de $G$ a le même nombre d'arêtes :

\begin{proposition}
Un graphe $G$ planaire, à $n \geq 3$ sommets, a au plus $3n - 6$ arêtes.
\end{proposition}

\begin{definition}
La \emph{maille (girth)} $g(G)$ d'un graphe $G$ est la longueur d'un plus petit cycle dans $G$, s'il en existe un. Sinon, $g(G)$ est infinie. %(Les arbres ont des mailles infinies).
\end{definition}

\subsection{Coloration de Graphes}
La coloration de graphes est un domaine central dans la théorie de graphes. 
%Notre problème s'agit d'un problème de coloration, donc cette partie sera consacrée à la définition des aspects sur la coloration.
Nous présentons ainsi les définitions et quelques propriétés des colorations des graphes.


\begin{definition}%[Coloration propre]
Une \emph{coloration propre} d'un graphe $G$ est une application $\varphi: V(G) \to C$, où $C=\{1,2,3,\dots,k\}$ est un ensemble d'entiers (de \emph{couleurs}), telle que pour tout arête $uv \in E(G)$, on a $\varphi(u) \neq \varphi(v)$. 

Un graphe $G$ est dit \emph{$k$-coloriable} si on peut le colorier à $k$ couleurs, c-à-d si l'ensemble $C$ de couleurs est de taille $k$. 

Le \emph{nombre chromatique} $\chi(G)$ d'un graphe $G$, est le plus petit entier $k$ tel que $G$ est $k$-coloriable. Si $\chi(G) = k$, alors on dit que le graphe G est \emph{$k$-chromatique}. Cela signifie donc que $G$ est $k$-coloriable mais  pas $(k-1)$-coloriable.
\end{definition}

%\begin{exemple}
%.... graphe et colorations
%\end{exemple}

%\begin{definition}
%Le problème de $k$-coloration est un problème de décision, qui vérifie si un graphe peut être colorié avec $k$ couleurs différentes. Soit donc, un graphe $G$ et un entier $k$ et on se demande si le graphe $G$ est $k$-coloriable.
%\end{definition}

Le problème de $k$-coloration (le problème de décider si un graphe donné est $k$-coloriable) est NP-complet pour tout $k \geq 3$ \cite{NP}. Pour $k = 1$, un graphe $G$ est $1$-coloriable si et seulement si $G$ ne contient pas d'arêtes. Puis, pour $k =  2$, on revient sur la question de tester si un graphe $G$ est biparti (ce qui est le cas si et seulement si il ne contient aucun cycle impair). 

Dans le domaine de la coloration de graphes il existe plusieurs types de colorations et même si la coloration propre est la plus connue ou la plus utilisée, dans notre problème on va s'intéresser à la coloration d'arêtes également.

\begin{definition}%[Coloration d'arêtes]
Une \emph{coloration d'arêtes} d'un graphe $G$ est une application $\varphi: E(G) \to C$, où $C=\{1,2,3,\dots,k\}$ est un ensemble d'entiers (de \emph{couleurs}), telle que pour tout paire $e, e' \in E(G)$ d'arêtes adjacentes on a $\varphi(e) \neq \varphi(e')$.

L'\emph{indice chromatique} $\chi'(G)$ d'un graphe $G$ est le nombre de couleurs minimum qu'on nécessite pour colorier les arêtes de $G$.
\end{definition}

\begin{theorem}[Vizing]
Soit $G$ un graphe. Alors, $\chi'(G) = \Delta(G)$ ou $\chi'(G) = \Delta(G) + 1$.
\end{theorem}

La question de déterminer/caractériser les classes de graphes pour lesquelles on a $\chi'(G) = \Delta(G)$ (on les appelle aussi les graphes de classe 1) reste une question centrale. 

Il existe des résultats concernant la classification des graphes de classe 1 pour les graphes planaires.
C'est Vizing \cite{Vizing} qui a montré que les graphes planaires avec $\Delta(G)\ge 8$ sont des graphes de classe 1. Le cas de $\Delta = 7$ a été prouvé en 2001 par Sanders et Zhao \cite{SandersZhao}. 
En revanche, pour $\Delta\le 5$ il existe des graphes planaires avec $\chi'(G)=\Delta(G)+1$ -- il suffit de subdiviser par un sommet de degré 2 une arête d'un graphe $\Delta$-régulier et $\Delta$-arête-coloriable \cite{Vizing2}.
Le cas $\Delta = 6$ reste largement ouvert. 

Cependant, les graphes planaires avec $\Delta(G)=5$ sans triangles (c-à-d avec $g(G)\ge 4$) sont 5-arête-coloriables, donc classe 1 ; pareil, les graphes planaires avec $\Delta(G) = 4$ et $g(G)\ge 5$ sont $4$-coloriables. Ces deux derniers cas, on les reprouvera ci-dessus. Il reste encore une classe de graphe planaire, plus riche que cette dernière, pour laquelle la question est toujours ouverte :

\begin{conjecture}
Soit $G$ un graphe planaire avec $\Delta(G) = 4$ et $g(G) \geq 4$. Alors, $\chi'(G) = 4$. 
\end{conjecture} 

Cette conjecture est le problème principal à traiter dans cet écrit. 
%\subsection{Sujet du Problème et Conditions}

\section{Réductibilité et déchargement}
\label{chap:easy}

Nous allons refaire la démonstration de deux propositions voisines de la conjecture étudiée, afin de pouvoir illustrer sur ces deux exemples-ci non-triviaux les méthodes utilisées pour démontrer ce type d'énoncé. Nous introduisons d'abord quelques notions et notation techniques.

Soit $G$ un graphe avec les arêtes coloriées. 
Considérons le sous-graphe $G[c_1,c_2]$ de $G$ induit par les arêtes coloriés $c_1$ et $c_2$. Par la définition d'une coloration, aucun sommet de $G[c_1,c_2]$ ne peut avoir un degré supérieur à 2. Par conséquent, toute composante connexe de $G[c_1,c_2]$ est une chaîne ou un cycle. Si on choisit une composante connexe $C$ de $G[c_1,c_2]$, qu'on échange les couleurs $c_1$ et $c_2$ le long de $C$ et qu'on garde la couleur de toutes les autres arêtes, on obtiendra une autre coloration de $G$.

Étant donné une arête $e$ de $G$ de couleur $c$ et une autre couleur $c'$, il existe une et une seule composante connexe de $G[c,c']$ contenant $e$.
Nous allons appeler celle-là la \emph{$(c,c')$-chaîne de Kempe} de $e$.


%Cette partie sera consacrée à la présentation des deux propositions qui ont été décrites précédemment. Ces deux propositions vont introduire la conjecture qu'on va traiter après.

\begin{proposition}
Soit $G$ un graphe planaire avec $\Delta(G) = 5$ et $g(G) \geq 4$. Alors $\chi'(G)=5$.
\label{prop:545}
\end{proposition}

Preuve. Supposons que la proposition est fausse. Soit $G$ un contre-exemple minimal, c-à-d un graphe planaire avec $\Delta = 5$ et $g \geq 4$ qui n'est pas $5$-arête-coloriable, mais tout sous-graphe de $G$ l'est.

D'abord, nous montrons quelques propriétés structurelles de $G$.

\begin{lemme}
Il n'existe pas de sommet $v$ de $G$ tel que $d_G(v) = 1$.
\label{le:1}
\end{lemme}

Preuve. Soit $v \in V(G)$ tel que $d_G(v) = 1$. Alors, il existe une $5$-arête-coloration de $G'=G \setminus v$. Soit $u$ le voisin de $v$. Par définition, $d_G(u) \leq 5$ donc $d_{G'}(u) \leq 4$. Par conséquence, il existe au moins une couleur $c$ avec laquelle aucune arête incidente à $u$ dans $G'$ n'est coloriée, donc on peut colorier l'arête $uv$ avec $c$. On tombe donc sur une contradiction avec la supposition que $G$ n'est pas $5$-arête-coloriable.
\ep

\begin{lemme}
Soit $v \in V(G)$ tel que $d_G(v) = 2$. Alors $\forall u \in N(v)$, $d_G(u) = 5$.
\label{le:2}
\end{lemme}

Preuve.
Supposons qu'un sommet $v \in V(G)$ tel que $d_G(v) = 2$ ait un voisin $u \in N(v)$ avec $d_G(u)  \le 4$. Considérons une 5-arête-coloration de $G' = G \setminus uv$.
L'arête $uv$ est adjacente à quatre arêtes de $G'$, alors, il existe au moins une couleur $c$ pour colorier l'arête $uv$, donc $G$ aussi est $5$-arête-coloriable,  une contradiction.
\ep


\begin{lemme}
Soit $v\in V(G)$ tel que $d_G(v) = 3$. Alors, $\forall u \in N(v)$, $4\le d_G(u) \le 5$.
\label{le:33}
\end{lemme}

Preuve. De la même manière que la démonstration du lemme précédent, si $v$ avait un voisin $u$ de degré (au plus) 3, l'arête $uv$ serait adjacente à (au plus) quatre autres arêtes, alors, n'importe quelle coloration de $G\setminus uv$ donnerait une coloration de $G$, une contradiction.

Nous savons ainsi que les voisins d'un sommet de degré 3 dans $G$ sont de degré 4 ou 5. On peut en dire encore plus.

\begin{lemme}
Soit $v\in V(G)$ tel que $d_G(v) = 3$. Alors, au moins deux voisins de $v$ sont de degré $5$.
\label{le:434}
\end{lemme}

Preuve.
Supposons, par l'absurde, que $N(v)=\{u_1,u_2,u_3\}$, où $4\le d_G(u_1) \le 5$ et  $d_G(u_2) = d_G(u_3) = 4$. 

Soit $e_i=vu_i$, $i=1,2,3$ ; soit $\{e_i,e_{i1},e_{i2},e_{i3}\}$ l'ensemble des arêtes incidentes à $u_i$, $i=2,3$. 
Considérons une coloration $\varphi$ du sous-graphe $G' = G \setminus e_3$. Afin que celle-ci ne s'étende pas en une coloration de $G$, on peut supposer, sans perdre de la généralité, que $\varphi(e_1) = 1$, $\varphi(e_2) = 2$, $\varphi(e_{31}) = 3$, $\varphi(e_{32}) = 4$ et $\varphi(e_{33}) = 5$.

Si les $(2,3)$-chaînes de Kempe des arêtes $e_2$ et $e_{31}$ sont distinctes, alors en alternant les couleurs d'une parmi elles on obtiendrait une coloration de $G'$ dans laquelle les arêtes adjacentes à $e_3$ ne seraient pas coloriées de cinq couleurs différentes, ce qui est absurde.
Donc, il existe une $(2,3)$-chaîne de Kempe reliant $e_2$ et $e_{31}$ ; il y a ainsi une arête incident à $u_2$ coloriée 3, disons que $\varphi(e_{21})=3$. Voir Figure \ref{fig:434} pour illustration.

Similairement, il existe une $(2,4)$-chaîne (une $(2,5)$-chaîne) reliant $e_2$ et $e_{32}$ ($e_2$ et $e_{33}$, respectivement). Par ailleurs, il n'y a pas d'arête coloriée 1 incidente à $u_2$.

Considérons maintenant les $(1,3)$-chaînes de Kempe. Par le même argument, la chaîne qui commence par $e_1$ se termine forcément par $e_{31}$. Donc, on est libre à alterner la chaîne de Kempe de $e_{21}$, recolorier $\varphi(e_2)=3$ et poser  
$\varphi(e_3)=2$, créant ainsi une coloration de $G$, une contradiction. 
\ep 
    
\begin{figure}[ht]
\centerline{
\begin{tabular}{ccc}
\begin{tabular}{c}
\includegraphics[scale=1]{./Figures/434.1}
\end{tabular}
&
$\longrightarrow$
&
\begin{tabular}{c}
\includegraphics[scale=1]{./Figures/434.2}
\end{tabular}
\end{tabular}
}
\caption{Un sommet de degré trois avec deux voisins de degré quatre constitue une configuration réductible. }
\label{fig:434}
\end{figure}

\begin{lemme}
Soit $uv \in E(G)$ tel que $d_G(u) = 2$ et $d_G(v) = 5$. Alors, tous les voisins de $v$ sauf $u$ sont de degré $5$. 
\label{le:254}
\end{lemme}

Preuve. 
Supposons le contraire, c-à-d, que $v$ ait un voisin $w$ de degré au plus 4.

Soit $e=uv$, soit $e'$ l'autre arête incidente à $u$.
Considérons une 5-arête-coloration $\varphi$ de $G'=G \setminus e$. On peut supposer que $\varphi(e')=1$ et que les quatre arêtes incidentes à $v$ dans $G'$ sont coloriées avec les couleurs $2,3,4,5$, dont $\varphi(vw)=2$ (Sinon, on trouverait une coloration de $G$ facilement).

Pour $i=2,3,4,5$, la $(1,i)$-chaîne de Kempe de $e'$ doit se terminer par une arête incidente à $v$, sinon, on aurait pu alterner la coloration de cette chaîne de sorte que l'arête $e$ puisse être coloriée avec la couleur 1. 

Par conséquent, il existe une arête incidente à $w$ coloriée 1, disons $f$. D'ailleurs, il existe une couleur, disons 3, avec laquelle aucune arête incidente à $w$ n'est coloriée (rappelons que $d_{G'}(w)=d_G(w)=4$). Donc, la $(1,3)$-chaîne de Kempe de $f$ est bien distincte de la $(1,3)$-chaîne de Kempe de $e'$.

Après avoir alterné les couleurs la $(1,3)$-chaîne de Kempe de $f$, en recoloriant $\varphi(vw)=1$ et en posant $\varphi(e)=2$ on obtient une coloration de $G$, une contradiction. \ep

\begin{lemme}
Soit $v \in V(G)$ tel que $d_G(v) = 5$. Alors, $v$ a au plus deux voisins de degré $3$.
\label{le:5333}
\end{lemme}


Preuve.
Supposons que cette condition n'est pas vrai, i.e., que $v$ ait trois voisins de degré 3, disons $u_1$, $u_2$ et $u_3$. Soit $u_4$ et $u_5$ les autres voisins de $v$. Soit $e_i = u_iv$ pour tout $i \in \{1,2,3,4,5\}$. Soit $\{e_i, e_{i1}, e_{i2}\}$ l'ensemble des arêtes incidentes à $u_i$ pour $i \in {1,2,3}$. Considérons une $5$-arête-coloration $\varphi$ de $G' = G \setminus e_1$. On peut supposer que $\varphi(e_2)=2$, $\varphi(e_3)=3$, $\varphi(e_4)=4$ et $\varphi(e_5)=5$, et que $\varphi(e_{11}) = 1$. 

On peut même supposer que $\varphi(e_{32}) \in \{3,5\}$.
Donc, pour $i \in \{2,4\}$, la $(1,i)$-chaîne de Kempe relie $e_{11}$ à $e_i$, ce qui provoque que $u_2$ est incident à une arête coloriée 1, disons $e_{21}$. 

Si $\varphi(e_{22}) \ne 4$, alors la $(1,4)$-chaîne de Kempe de $e_{21}$ est bien différente de celle reliant $e_{11}$ et $e_4$. En alternant les couleurs de la $(1,4)$-chaîne de Kempe de $e_{21}$, on trouverait une coloration $\varphi'$ avec $\varphi'(e_{21})=4$, ce qui permettrait de recolorier $\varphi'(e_2) = 1$ laissant disponible la couleur $2$ pour l'arête $e_1$. On obtiendrait donc une coloration de $G$, ce qui est absurde. 

Par conséquence, $\varphi(e_{22}) = 4$. Colorions $\varphi(e_1)=2$, recolorions $\varphi(e_2)=3$ et enlevons la couleurs de $e_3$.
On peut supposer que $\varphi_{31}=1$, sinon, la couleur $1$ serait libre pour l'arête $e_3$.

En considérant les $(1,2)$-chaînes de Kempe, on peut déduire que $\varphi(e_{32})=2$. Or, dans ce cas-là, la $(1,4)$-chaîne de Kempe de $e_{31}$ est bien différente de celle reliant $e_{11}$ et $e_4$, que cette dernière passe par $u_2$ ou pas. On peut donc transformer $\varphi$ en une autre coloration avec $\varphi(e_{31}=4)$ et colorier $e_3$ de la couleur 1, ce qui est absurde.
\ep


\bigskip
Ensuite, on va montrer que les propriétés structurelles de $G$ montrées ci-dessus entraînent une contradiction à l'aide de la technique de déchargement. Cette technique a été utilisée aussi pour prouver le théorème de quatre couleurs. 

%Pour la méthode de déchargement, premièrement une condition est définie. 
Soit $w: V(G) \to \mathbb{R}$  la fonction définie par
 $$
 \gathered
 w(v) = d_G(v) - 4 \qquad \textrm{pour tout $v\in V(G)$,} \\
 w(f) = d_G(f) - 4 \qquad \textrm{pour tout $f\in F(G)$,} 
\endgathered 
$$
appelée aussi la \emph{charge initiale} des sommets et des faces de $G$. Nous montrons d'abord que la somme totale de la charge dans $G$ est strictement négative, puis nous définissons des règles de redistribution de charge telles que la somme totale de la charge reste invariante, finalement, nous observons qu'après la redistribution il ne reste plus de charge strictement négative, ce qui est absurde.

\begin{lemme}
Soit $G$ un graphe planaire connexe. Alors,
$$ \sum_{v \in V(G)} (d_G(v) - 4) + \sum_{f\in F} (d_G(f) -4) = -8.$$
\label{le:charge}
\end{lemme}

Preuve.
D'après la formule d'Euler et la formule de poignées de mains, on a
$$ 
n - m + f = 2, \qquad
\sum_{v\in V(G)} d_G(v) = 2m, \qquad
\sum_{f\in F} d_G(f) = 2m. 
$$

Alors,

$$ \sum_{v \in V(G)} (d_G(v) - 4) + \sum_{f\in F} (d_G(f) -4) = 2m -4n + 2m - 4f = -4(n - m + f) = -4\cdot 2 = -8.$$
\ep 

Selon la définition de la charge initiale, on sait que
$$ 
\gathered
w(v) = -2 \qquad \textrm{pour tout $v \in V(G)$, tel que } d_G(v) = 2, \\
w(v) = -1 \qquad \textrm{pour tout $v \in V(G)$, tel que } d_G(v) = 3, \\
w(v) =  0 \qquad \textrm{pour tout $v \in V(G)$, tel que } d_G(v) = 4, \\
w(v) =  1 \qquad \textrm{pour tout $v \in V(G)$, tel que } d_G(v) = 5.
\endgathered
$$

Par ailleurs, la charge initiale de toutes les faces est positive puisque $g(G)\ge 4$.

Nous redistribuons la charge selon les règles suivantes:
\begin{enumerate}
\item[(R1)] Tout sommet de degré 5 donne une unité de charge à chacun de ses voisins de degré 2.
\item[(R2)] Tout sommet de degré 5 donne $1/2$ de charge à chaque voisin de degré 3.
\end{enumerate}

Évidemment, la somme totale de charge reste invariante et la charge de toutes les faces reste positive.

Vérifions que pour toute sommet de $G$ la charge finale est positive également, ce qui nous donnera la contradiction.
Selon Lemme \ref{le:1} il n'y a pas de sommet de degré 1.
Les sommets de degré 4 ni donnent ni reçoivent rien, leur charge reste nulle.

Soit $v$ un sommet de degré 2. Selon Lemme \ref{le:2}, il a deux voisins de degré 5. D'après (R1), sa charge finale est de $-2 + 2\cdot 1=0$.

Soit $v$ un sommet de degré 3. Selon Lemme \ref{le:434}, il a au moins deux voisins de degré 5. D'après (R2), sa charge finale est au moins $-1 + 2\cdot 1/2 = 0$.

Soit $v$ un sommet de degré 5. Si $v$ a un voisin de degré 2, selon Lemme \ref{le:254} il n'a pas d'autre voisins de degré 2 ou 3, donc, il ne donne que de la charge de son (unique) voisin de degré 2. Sa charge finale est donc de $1-1=0$.
Supposons alors que $v$ n'ait pas de voisin de degré 2. Si $v$ a des voisins de degré 3, il en a au plus deux, grâce à Lemme \ref{le:5333}. Sa charge finale est alors d'au moins $1-2\cdot 1/2 = 0$. 

Alors, après la redistribution de charge on obtient un graphe chargé non-négativement, ce qui est absurde. On confirme alors l'hypothèse de base, celle qui dit que le graphe est $5$-arête-coloriable.
%Ces règles sur un graphe qui respecte les propriétés structurelles définies précédemment, après la redistribution de charge, il reste positif ou nul, ce qui provoque une contradiction sur la proposition initiale.
\ep


%De cette façon on prouve la conjecture pour les graphes planaires de degré maximum $\Delta = 5$ sans triangles. Ensuite on va faire la même procédure pour les graphes planaires de degré maximum $\Delta = 4$, pour une maille de taille 5 d'abord et puis pour une maille de taille 4.

\begin{proposition}
Soit $G$ un graphe planaire avec $\Delta(G) = 4$ et $g(G) \geq 5$. Alors $G$ est $4$-arête-coloriable.
\label{prop:454}
\end{proposition}

Preuve. Supposons que la proposition est fausse. Soit $G$ un contre-exemple minimal. Dans ce cas aussi nous allons d'abord montrer des propriétés structurelles de $G$. Les deux premières sont exactement les mêmes que pour la proposition précédente, nous omettons la preuve.

\begin{lemme}
Il n'existe pas de sommet $v$ de $G$ tel que $d_G(v) = 1$.
\label{le:14}
\end{lemme}

\begin{lemme}
Soit $v \in V(G)$ tel que $d_G(v) = 2$. Alors $\forall u \in N(v)$, $d_G(u) = 4$.
\label{le:23}
\end{lemme} 


\begin{lemme}
Soit $uv \in E(G)$ tel que $d_G(u) = 2$ et $d_G(v) = 4$. Alors, tous les voisins de $v$ sauf $u$ sont de degré $4$.
\label{le:243}
\end{lemme}

Preuve. Nous suivons les idées de la preuve de Lemme \ref{le:254}.
Supposons par l'absurde que $v$ ait un voisin $w$ tel que $d_G(w) = 3$. Soit $e = uv$ et $e'$ l'autre arête incidente de $u$. Considérons une $4$-arête-coloration $\varphi$ de $G' = G \setminus e$. On peut supposer que $\varphi(e') = 1$ et que les trois arêtes incidentes de $v$ dans $G'$ sont coloriées avec les couleurs 2, 3 et 4, dont $\varphi(vw) = 2$.

Il existe forcément une $(1,i)$-chaîne de Kempe reliant $e'$ à une arête incidente à $v$ dans $G'$ pour tout $i \in \{2,3,4\}$. Par conséquence, il existe une arête incidente à $w$ coloriée avec la couleur 1, disons $e''$. D'ailleurs, il existe une couleur $c\in \{3,4\}$ que $w$ ne voit pas. On peut donc renverser la couleur 1 et $c$ obtenant $\varphi(e'') = c$ ce qui permet de colorier $\varphi(vw) = 1$ et $\varphi(e) = 2$. On trouve alors une 4-arête-coloration de $G$ ce qui est absurde.
\ep

\begin{lemme}
Soit $v \in V(G)$ tel que $d_G(v)=3$. Alors, $v$ a au moins deux voisins de degré $4$.
\label{le:333}
\end{lemme}

Preuve.
Supposons qu'il existe $u_1$, $u_2 \in N(v)$ tel que $d(u_1) = d(u_2) = 3$. Soient $e_1 = u_1v$, $e_2 = u_2v$ et $e_3$ les arêtes incidentes à $v$. Notons $e_{ij}$ les arêtes incidentes à $u_i$ différentes de $e_i$, pour $i \in \{1,2\}$, $j \in \{1,2\}$. Considérons une $4$-arête-coloration de $G' = G \setminus e_1$ qui ne soit pas une coloration de $G$. Alors, on peut supposer que  $\varphi(e_2)=1$, $\varphi(e_3)=2$,$\varphi(e_{11})=3$ et $\varphi(e_{12})=4$. De plus, il existe forcément des $(i,j)$-chaînes de Kempe pour $i \in {1,2}$ et $j \in {3,4}$ (sinon il serait facile de trouver une $4$-arête-coloration de $G$) reliant les arêtes incidentes à $v$ avec les arêtes incidentes à $u_1$ dans $G'$. On peut donc supposer que $\varphi(e_{21})=3$ et $\varphi(e_{22})=4$.

Observons maintenant que la $(2,4)$-chaîne de Kempe reliant $e_{12}$ et $e_2$ est bien différente de la $(2,4)$-chaîne de Kempe de $e_{22}$. Alors on peut trouver, en renversant les couleurs de cette dernière chaîne, une coloration de $G'$ telle que $\varphi(e_{22}) = 2$, ce qui permet de poser $\varphi(e_2) = 4$ tout en laissant la possibilité de colorier $e_1$ avec la couleur 2. On obtient donc une $4$-arête-coloration de $G$ ce qui provoque une contradiction.
\ep


Les propriétés que nous venons de vérifier vont entraîner une contradiction grâce à la procédure de déchargement. Similairement que dans la proposition précédente, on définit d'abord la charge initiale par la fonction suivante :
$$
\gathered
w(v) = \frac{3}{2} d_G(v) - 5 \qquad \textrm{pour tout $v\in V(G)$,} \\
w(f) = d_G(f) - 5 \qquad \textrm{pour tout $f\in F(G)$.}
\endgathered
$$

Puis, nous montrons que la somme de la charge totale est strictement négative, et que après la redistribution de la charge par des règles de déchargement, la charge de tous les éléments du graphe deviendra positive, ce qui provoquera une contradiction.


\begin{lemme}
Soit $G$ un graphe planaire connexe. Alors,
$$
\sum_{v \in V(G)} \left(\frac{3}{2} d_G(v) - 5\right) + \sum_{f\in F(G)} \left(d_G(f) -5\right) = -10.
$$
\label{le:cha2}
\end{lemme}

Preuve.
Tout comme pour Lemme \ref{le:charge}, on déduit de la formule d'Euler que
$$ 
\sum_{v \in V(G)} \left(\frac{3}{2}d_G(v) - 5\right) + \sum_{f\in F(G)} \left(d_G(f) -5\right) = 3m -5n + 2m - 5f = -5(n - m + f) = -5 \cdot 2 = -10.
$$
\ep 

Connaissant la charge des sommets, on sait que,
$$
\gathered
w(v) = -2 \qquad \textrm{pour tout $v \in V(G)$, tel que } d_G(v) = 2,\\
w(v) = -\frac{1}{2} \qquad \textrm{pour tout $v \in V(G)$, tel que } d_G(v) = 3,\\
w(v) =  1 \qquad \textrm{pour tout $v \in V(G)$, tel que } d_G(v) = 4.
\endgathered
$$

D'ailleurs, la charge initiale de toutes les faces est positive puisque $g(G) \geq 5$.

Alors, les règles qui vont distribuer la charge des sommets sont les suivantes :

\begin{enumerate}
\item[(R1)] tout sommet de degré 4 donne $1$ de charge à chacun de ses voisins de degré 2.
\item[(R2)] tout sommet de degré 4 donne $1/4$ de charge à chaque voisin de degré 3.
\end{enumerate}
Dans ce cas là aussi, la somme totale de la charge reste invariante et la charge de toutes les faces reste positive.

On va vérifier que pour tout sommet de $G$ la charge finale est positive, ce qui donnera la contradiction. On a déjà montré l'inexistence des sommets de degré 1 dans ces configurations grâce au Lemme \ref{le:14}. Puis, selon Lemme \ref{le:23}, tout sommet de degré 2 n'a que des voisins de degré 4, donc par (R1) la charge finale d'un sommet de degré 2 reste nulle.

Ensuite, soit $v$ un sommet de degré $3$. Selon Lemme \ref{le:333}, le sommet $v$ a au moins deux voisins de degré $4$. Par (R2) on sait que $v$ reçoit 1/4 d'unité de charge négative de chacun de ses deux voisins de degré $4$. Sa charge est donc d'au moins $-1/2+2\cdot 1/4 = 0$.

%Finalement, ces sommets adjacentes aux sommets de degré deux on sait qu'ils ne reçoivent plus de charge de ses autres voisins car selon  Lemme \ref{le:243}, tout sommet adjacent d'un sommet de degré 2 n'a que des voisins de degré 4 (sauf le voisin de degré 2). Alors, d'après (R1) l'endroit autour d'un sommet de degré 2 reste avec une charge finale positive.

Finalement, soit $v$ un sommet de degré 4. Si $v$ a un voisin de degré 2, selon le Lemme \ref{le:243} tous ses autres voisins sont de degré 4, alors, sa charge est de $1-1=0$. Si $v$ a des voisins de degré 3, on sait qu'il ne peut avoir que 4 voisins de degré 3, donc il peut donner au maximum $4 \cdot 1/4=1$ de charge à ces voisins. Alors la charge finale de $v$ est au moins $1 - 4\cdot 1/4 = 0$. Dans tous les cas on obtient une somme de la charge totale positive.

Alors, après la redistribution de charge on obtient un graphe chargé non-négativement, ce qui est absurde. On confirme alors l'hypothèse de base, celle qui dit que le graphe est $5$-arête-coloriable.
\ep 


\subsection{Théorème de 4 Couleurs}

Le Théorème de 4 couleurs est un des théorèmes le plus connu dans le domaine de la coloration des graphes et même de la théorie des graphes. Le théorème dit que n'importe quel graphe planaire est coloriable avec 4 couleurs. C'est une conjecture proposé par Gurthrie en 1852 et qui a été prouvé par Appel et Haken \cite{A&H} en 1976. Malheureusement, la preuve de Appel et Haken n'a pas été complètement accepté car il y avait une partie faite sur machine difficile à vérifier et une autre partie dont l'explication n'était pas tout à fait précise.

C'est pour ce fait-ci qu'il y a eu plusieurs personnes qui ont tenté de démontrer ce théorème, et c'est Robertson, Sanders, Seymour et Thomas \cite{T4C} qui l'ont ré-prouvé en 1995. En fait, dans cette nouvelle preuve ils se sont inspirés largement de la preuve de Appel et Haken, cependant, ils ont simplifié d'une manière significative la procédure de déchargement, et leur programme de vérification de réductibilité automatisée est plus robuste et indépendant du déchargement.
%t ils ont utilisé une méthode de réduction pour trouver des contra-exemples minimales pour toutes les configurations pour lequel si on trouve une configuration qui est \emph{réductible} on ne regarde pas cette configuration parce qu'on regarde sa \emph{réduction} ou partie plus petite. Puis il faut montrer que aucun des contre-exemples contredit l'hypothèse. Pour ce fait ils ont utilisé la méthode de déchargement. 

\begin{theorem}[Appel, Haken]
Soit $G$ un graphe planaire. Alors $G$ est 4-coloriable.
\label{th:4CT}
\end{theorem}

Si un graphe $G$ est $k$-coloriable, alors tout sous-graphe $H$ de $G$ l'est également. C'est pour cela qu'il a suffit de démontrer Théorème \ref{th:4CT} pour les graphes planaires maximales -- les triangulations.


%La démonstration du Théorème \ref{th:4CT} s'appuie sur la proposition suivante, facile à vérifier.

%\begin{proposition}
%Une triangulation est 4-coloriable ssi le graphe dual est 3-arête-coloriable.
%\end{proposition}

La preuve de Théorème \ref{th:4CT} suit les idées suivantes : Selon la formule d'Euler, le degré moyen d'une triangulation est inférieur à 6, donc, il y a des sommets de degré au plus 5 dans le graphe. Supposons le théorème faux. Il existe alors des contre-exemples, considérons en le plus petit. 

Il est montré qu'il existe un ensemble $\mathcal{S}$ de 633 configurations contenant des sommets de degré 5 (et éventuellement d'autres degrés) tel que toute triangulation comporte un sous-graphe isomorphe à une des configurations de $\mathcal{S}$. Pour le mettre en évidence, une procédure de déchargement est introduite : Tout sommet d'une triangulation est assigné une charge initiale de façon que la charge totale du graphe est négative. La charge est redistribuée parmi les sommets voisins suivant des règles soigneusement définies afin que les charges positives et négatives s'éliminent entre elles. Il est démontré (à l'aide d'un ordinateur) que tout sommet avec une charge finale négative fait partie d'une des configurations de $\mathcal{S}$.

Pour compléter la preuve, il faut encore vérifier que toute configuration $C$ de $\mathcal{S}$ est \emph{réductible}, c-à-d, si un graphe $G$ comporte $C$ comme sous-graphe, alors toute coloration du graphe $G'$ obtenu en remplaçant $C$ par un autre sous-graphe plus petit s'étend en une coloration de $G$, ce qui entraînerait une contradiction avec la supposition que $G$ soit le contre-exemple le plus petit.
La réductibilité des 633 configurations de $\mathcal{S}$ est vérifiée à l'aide d'un ordinateur également.






\section{Réductibilité}
\label{chap:red}
Les démonstrations de Lemmes \ref{le:1}-\ref{le:5333} et \ref{le:14}-\ref{le:333} ont toutes la même structure: La supposition que le contre-exemple minimal $G$ contienne un arrangement de sommets de certains degrés entraîne une contradiction avec non-colorabilité de $G$.
  

Nous allons généraliser cette notion.

\begin{definition}
Soit $H$ un graphe étiqueté, c-à-d un graphe $H= (V(H), E(H))$ avec une fonction $\gamma$: $V(G) \to \mathbb{N}$ telle que $\forall v \in V(H)$ on a $\gamma(v) \geq d_H(v)$. Le graphe $H$ est une \emph{configuration} dans $G$ s'il existe un sous-graphe $X$ de $G$ et un isomorphisme $\xi$: $H \to X$ tel que $\forall v \in V(H)$, $\gamma(v) = d_G(\xi(v))$. 
\label{de:conf}
\end{definition}

Dans la suite, lorsque nous parlons d'une configuration $H$ dans un graphe $G$, nous identifions $H$ et sa copie $X=\xi(H)$ dans $G$.

\begin{definition}
Soit $H$ une configuration de $G$. Notons
$$
\partial_G(H) = \{e =uv \in E(G) : u \in V(H)\textrm{ et }v \in V(G \setminus H)\},
$$ l'ensemble des arêtes sortantes de $H$, et $r=|\partial_G(H)|$ leur nombre.
Notons $H^* = H \cup \partial_G(H)$ et $G'=G\setminus H \cup \partial_G(H)$.
\label{de:ring}
\end{definition}

Si $\varphi:E(G)\to \{1,2,3,4\}$ est une 4-arête-coloration de $G$, alors $\varphi$ induit une coloration de $\partial_G(H)$, de $H^*$, et de $G'$. Pour capturer la structure de la coloration $\varphi$ dans $G'$, nous introduisons les notions et notations suivantes :

\begin{definition}
Soit $G$ un graphe avec $\Delta(G)=4$ et $H$ une configuration de $G$.
Soit $\varphi:E(G')\to \{1,2,3,4\}$ une 4-arête-coloration de $G'$. 
Soient $c, c' \in \{1,2,3,4\}$, $c\ne c'$ deux couleurs quelconques. 
Une coloration $\varphi':E(G)\setminus E(H)\to \{1,2,3,4\}$ est \emph{adjacente} à $\varphi$, si $\varphi'$ peut être obtenu à partir de $\varphi$ en alternant les couleurs d'une $(c,c')$-chaîne de Kempe de $e$ pour un choix d'une arête $e\in\partial_G(H)$ et des couleurs $c,c'\in\{1,2,3,4\}$. Notons $N(\varphi)$ l'ensemble des colorations adjacentes à $\varphi$.
\end{definition}

\begin{definition}
Soit $G$ un graphe avec $\Delta(G)=4$ et $H$ une configuration de $G$.
Soit $\varphi:E(G')\to \{1,2,3,4\}$ une 4-arête-coloration de $G'$. 
Soient $c, c' \in \{1,2,3,4\}$, $c\ne c'$ deux couleurs quelconques. 
Le \emph{graphe résiduel}
$\mathcal{X}^{\varphi}_{c,c'}$ est défini par
\begin{itemize}
\item $V(\mathcal{X}^{\varphi}_{c,c'}) = \partial_G(H)$, 
\item $E(\mathcal{X}^{\varphi}_{c,c'}) = \{ e_ie_j$ tels que $(c,c')$-chaîne de Kempe de $e_i$ est la même que la $(c,c')$-chaîne de Kempe de $e_j\}$.
\end{itemize}
\end{definition}

Il est claire que $d_{\mathcal{X}_{\varphi}^{c,c'}}(e)\in\{0,1\}$ pour toute $e\in \partial_G(H)$. Le graphe résiduel est donc un couplage sur $\partial_G(H)$. Plus précisément, si un plongement de $G$ dans le plan est fixé et que les sommets du graphe résiduel sont placé de manière de respecter l'ordre circulaire des arêtes de $\partial_H(G)$, alors les arêtes de $\mathcal{X}_{\varphi}^{c,c'}$ ne se croisent pas, puisque les chaînes de Kempe ne se croisent pas non plus dans $G'$. Le graphe résiduel est donc un couplage planaire de $\partial_G(H)$.

\begin{proposition}
Soit $G$ un graphe avec $\Delta(G)=4$ et $H$ une configuration de $G$.
Soit $\varphi:E(G')\to \{1,2,3,4\}$ une 4-arête-coloration de $G'$. 
Soit $\varphi' \in N(\varphi)$ une coloration adjacente à $\varphi$ obtenu en alternant les couleurs de la $(c,c')$-chaîne de Kempe de $e\in \partial_G(H)$ pour $c,c'\in\{1,2,3,4\}$.
Soient $\beta = \varphi|_{\partial_G(H)}$ et $\beta' = \varphi'|_{\partial_G(H)}$ les colorations de  $\partial_G(H)$ induites par $\varphi$ et $\varphi'$. Alors,
$$
|\{e\in \partial_H(G) : \beta(e)\ne \beta'(e)\}| \in \{1,2\},
$$
de plus, on a \\
$|\{e\in \partial_H(G) : \beta(e)\ne \beta'(e)\}|=1$ si et seulement si $d_{\mathcal{X}_{\varphi}^{c,c'}}(e)=0$, et \\
$|\{e\in \partial_H(G) : \beta(e)\ne \beta'(e)\}|=1$ si et seulement si $d_{\mathcal{X}_{\varphi}^{c,c'}}(e)=1$. 
\end{proposition}

Preuve. Il est facile à observer que la modification de couleurs de la $(c,c')$-chaîne de Kempe de $e$ comporte un changement de couleur de $e$. Cela peut entraîne un changement de couleur d'une autre arête $e'$  de $\partial_G(H)$ si et seulement si cette chaîne-ci contient cette arête-là. \ep

%À l'aide de ces notions auxiliaires, nous pouvons définir l'ensemble de colorations de $\partial_G(H)$ telles que s'il existe une coloration de $G\setminus H$, alors il existe une coloration de $G$.

Considérons maintenant les colorations de $H$.
  
\begin{definition}
Soit $G$ un graphe avec $\Delta(G)=4$ et $H$ une configuration de $G$.
Soit $\beta$ une coloration de $\partial_G(H)$. Soit $c,c'\in \{1,2,3,4\}$ deux couleurs quelconques. Soit $X$ un couplage planaire de $\partial_G(H)$ tel que si $\beta(e)\notin \{c,c'\}$, alors $d_X(e)=0$. Une coloration $\beta'$ est \emph{adjacente} à $\beta$ par rapport à $c,c'$ et $X$ si soit
\begin{itemize}
\item $\beta'$ est obtenu à partir de $\beta$ en alternant la couleur (modifiant $c$ en $c'$ et vice versa) de deux arêtes $e,e'\in \partial_H(G)$ telles que $ee'\in E(X)$, soit
\item $\beta'$ est obtenu à partir de $\beta$ en alternant la couleur d'une arête $e$ telle que $d_e(X)=0$ et $\beta(e)\in\{c,c'\}$.
\end{itemize}
Nous notons $N^X_{c,c'}(\beta)$ l'ensemble de colorations adjacentes à $\beta$ par rapport à $c,c'$ et $X$.
\end{definition}

\begin{definition}
Soit $H$ une configuration de $G$ et $\beta$ une coloration de $\partial_G(H)$. Alors, $\beta$ \emph{s'étend} vers $H$ s'il existe une coloration $\psi$ de $H^*$ telle que $\psi|_{\partial_G(H)} = \beta$. Notons $\Phi_0$ l'ensemble des colorations de $\partial_G(H)$ qui s'étendent vers $H$, i.e.,
$$
\Psi_0 = \{\beta : \exists \psi : E(H^*)\to\{1,2,3,4\} \textrm{ une coloration telle que } \psi|_{\partial_G(H)} = \beta\}.
$$
Ensuite, pour $i\in\mathbb{N}$, soit
$$
\Psi_{i+1} = 
\{ 
\beta : 
\beta\in \Psi_i
\textrm{ ou }
\exists c,c'\in \{1,2,3,4\},\, 
\forall X\textrm{ couplage planaire de $\partial_G(H)$, }
N^X_{c,c'}(\beta) \cap \Psi_i \neq \emptyset 
\},
$$
c-à-d, l'ensemble des colorations $\beta$ telles que pour une paire de couleurs $c,c'$, quelque soit un couplage planaire $X$ de $\partial_G(H)$, il existe une coloration adjacente à $\beta$ déjà dans $\Psi_{i}$. Finalement, soit
 $$
 \Psi(H)=\bigcup_{i\in \mathbb{N}} \Psi_i(H).
$$
\end{definition}

\begin{theorem}
Soit $H$ une configuration de $G$, soit $H'$ une autre configuration telle que $\partial(H)=\partial(H')$. Soit $G'$ le graphe obtenu à partir de $G$ en remplaçant $H$ par $H'$.
Si $\Psi_0(H')\subseteq \Psi(H)$, alors, pour toute 4-arête-coloration $\varphi'$ de $G'$ il existe une 4-arête-coloration $\varphi$ de $G$.
\label{th:red}
\end{theorem}

Preuve. Soit $\varphi'$ une coloration de $G'$, soit $\beta'=\varphi'|\partial_H(G)$. Comme $\varphi$ est une coloration de $G'$, $\beta'$ est une coloration de $\partial_G(H')$ qui s'étend vers $H'$, d'où $\beta'\in \Psi_0(H')$.
Alors, il existe $i\in \mathbb{N}$ tel que $\beta' \in \Psi_{i}(H)$.
Nous démontrons l'existence d'une coloration $\varphi$ de $G$ par récurrence par rapport à $i$.

Si $i=0$, alors $\beta'$ s'étend vers $H$, c-à-d, il existe une coloration $\psi$ de $H^*$ telle que $\psi|\partial_G(H)=\beta'$. On définit
$$
\varphi (e) = 
\begin{cases}
\varphi'(e) & \text{si }e\notin H, \\
\psi(e) & \text{si } e\in H.
\end{cases}
$$

Il est évident que c'est une 4-arête-coloration de $G$

Supposons que $\beta' \in \Psi_{i+1}(H)\setminus \Psi_i(H)$ pour un $i\in \mathbb{N}$. Alors, il existe deux couleurs $c,c'$ telles que, quelque soit un couplage planaire $X$ de $\partial_G(H)$, il existe $\beta''\in \Psi_i(H)$ adjacente à $\beta'$. Considérons maintenant le couplage planaire $\mathcal{X}_{\varphi'}^{c,c'}$. La coloration $\beta''$ adjacente à $\beta'$ correspondante est une restriction d'une coloration de $G'$ adjacente à $\varphi'$, disons $\varphi''$. Puisque $\beta''\in \Psi_i(H)$, par l'hypothèse de récurrence, pour $\varphi''$ il existe une 4-arête-coloration $\varphi$ de $G$.

Autrement dit, il suffit de modifier la coloration $\varphi'$ le long des chaînes de Kempe un nombre fini de fois afin d'obtenir une coloration qui s'étend directement vers $H$. \ep

\begin{definition}
Soit $H$ et $H'$ deux configurations telles que $\partial(H)=\partial(H')$. La configuration $H$ \emph{se réduit} vers $H'$ si $\Psi_0(H')\subseteq \Psi(H)$. La configuration $H$ est \emph{réductible} s'il existe une configuration $H'$ vers la quelle elle se réduit.
\end{definition}

Théorème \ref{th:red} permet d'automatiser la vérification de réductibilité de configurations -- on n'a pas besoin de connaître la structure du graphe $G$, il suffit de deviner la réduction (la configuration $H'$) et vérifier que la condition de la définition dé réductibilité est satisfaite.

Dans le chapitre \ref{chap:Des}, nous donnons la description plus détaillée de l'implémentation de cette procédure.
%Si une configuration $H$ se réduit vers $H'$, dans un graphe on peut remplacer $H$ par $H'$, obtenant un contre-exemple minimal de $G$. Comme on a fait dans le chapitre précédente (Chapitre \ref{chap:easy}) avec les deux conjectures qui ont introduit ce cas, c'est nécessaire travailler avec des contre-exemples minimaux. Ces propriétés vont entraîner une contradiction grâce à la méthode de déchargement qui sera présentée lors du chapitre suivante. Puis on verra dans la partie des résultats les configurations qu'on a pu réduire et on analysera plus en profondeur la distribution de la charge par tout le graphe.


\section{Déchargement}
\label{chap:Dec}
Dans ce chapitre, nous présentons la charge initiale et des règles de déchargement qui permettraient de démontrer la conjecture, si nous complétions la liste de configurations réductibles. 
Nous avons déjà introduit le processus de déchargement dans le chapitre \ref{chap:easy}, dans ce cas-là nous allons faire pareil. 

Soit $G$ un graphe planaire avec $\Delta(G)=4$ et $g(G)\ge 4$.
La charge initiale $w$ des sommets et des faces de $G$ est définie par
$$
\gathered
w: V(G)\cup F(G) \to \mathbb{R}, \\
w(x) = d_G(x) - 4. \\
\endgathered
$$

Comme dans la démonstration du Proposition \ref{prop:545}, la somme de charge initial est égale à $-8$.

Notons que la charge de tous le sommets est négative et la charge de toutes les faces est positive. Plus précisément, 
$$
\gathered
w(v) = -2 \qquad \textrm{pour tout $v \in V(G)$, tel que  } d_G(v) = 2,\\
w(v) = -1 \qquad \textrm{pour tout $v \in V(G)$, tel que  } d_G(v) = 3,\\
w(v) = 0 \qquad \textrm{pour tout $v \in V(G)$, tel que  } d_G(v) = 4
\endgathered
$$
et
$$
\gathered
w(f) = 0 \qquad \textrm{pour tout $f \in F$, tel que  } d_G(f) = 4,\\
w(f) = 1 \qquad \textrm{pour tout $f \in F$, tel que  } d_G(f) = 5,\\
w(f) \geq 2 \qquad \textrm{pour tout $f \in F$, tel que  } d_G(f) \geq 6.
\endgathered
$$

Contrairement au cas des propositions \ref{prop:545} et \ref{prop:454}, il ne suffit pas de définir les règles de déchargement parmi les sommets, puisqu'il n'y a pas de sommet de charge strictement positive. Nous sommes forcés à utiliser la charge des faces de taille au moins 5 pour éliminer la charge négative des sommets de degré 2 et 3. 

Nous proposons la règle de déchargement provisionnelle suivante :
\begin{enumerate}
\item [(R)] Toute face de taille au moins 5 donne une unité de charge à chaque sommet incident de degré 2 ou 3.
\end{enumerate}

Dans la suite, il faut montrer que tout élément de graphe (sommet ou face) dont la charge reste strictement négative après la redistribution fait partie d'une configuration réductible.

Dans le chapitre \ref{chap:easy} nous avons déjà montré quelques configurations réductibles qui s'appliquent ici aussi. Selon Lemmes \ref{le:14}-\ref{le:333} nous savons déjà qu'il n'y a aucun sommet de degré 1, qu'un sommet de degré 2 n'a que les voisins de degré 4, que pour $v$ ayant un voisin de degré deux, les autres voisins de $v$ sont de degré 4, et qu'un sommet de degré 3 a au moins deux voisins de degré 4.

Soit $v$ un sommet de degré 2. Afin que sa charge devienne positive, il est nécessaire que toutes les deux faces incidentes à $v$ soient de taille au moins 5. Donc, il va falloir trouver des configurations réductibles avec un sommet de degré 2 incident à une face de taille 4.

Soit $v$ un sommet de degré 3. Afin que sa charge devienne positive, il est nécessaire qu'il soit incident à au moins une face de taille au moins 5. Donc, il va falloir trouver des configurations réductibles avec un sommet de degré 3 incident à que des faces de taille 4.

Les sommets de degré 4 et les faces de taille 4 ne participent pas au déchargement, leur charge reste nulle.

Soit $f$ une face de taille 5. Si $f$ est incidente à un sommet de degré 2, selon Lemme \ref{le:243} tous les autres sommets incidents à $f$ sont de degré 4, donc la charge de $f$ est $1-1=0$. Il reste à éliminer les cas où $f$ est incidente à plusieurs sommets de degré 3.

Pareil pour les faces de taille supérieur à 5, il va falloir trouver des configurations réductibles ... je vais manger.

%Il est essentiel de trouver aussi des configurations réductible contenant les sommets avec charge strictement négative (sommets de degré 2 et 3) ainsi que les faces sans charge strictement positive (des carrés). Il va falloir trouver également les configurations permettant d'éliminer le cas où une face de taille $k$ est incident à $\ell$ sommets de degré 2 ou 3, où $\ell \ge k-4$.

À partir de ce point, regardons quelles configurations seraient vont permettre que le graphe se décharge correctement. 


%D'ailleurs, soit $v$ un sommet de degré 3. Selon le Lemme \ref{le:333} $v$ ne peut avoir que un seul voisin de degré 3, alors considérons les deux cas. D'abord considérons l'existence d'un sommet voisin $u$ de degré 3. Pour que la charge totale de l'environnement de l'arête $e=uv$ devient positive (par la règle (R2)) après la distribution de charge c'est nécessaire qu'il ait deux faces autour de $e$ de taille au moins 5 ou dans le cas échéant, une face de taille au moins 6. C'est à dire, c'est nécessaire trouver une réduction pour une configuration avec 3 carrés et un pentagone autour de $e$. 

%Dans l'autre cas, c-à-d, si $v$ n'a que des voisins de degré 4, la condition pour que la configuration ait une charge total positive est, selon (R2), qu'une des faces autour de $v$ ait une taille au moins 5.

%Finalement, soit $v$ un sommet de degré 4. $v$ il n'a pas de charge à distribuer donc, supposons que les sommets de degré 4 ils sont autour de toutes les faces incidentes aux sommets de degré inférieur pour que pour toute configuration possible d'un graphe $G$, obtient, d'après la redistribution, une charge totale positive, ce qui concerne une contradiction.





%On a déjà montré que dans $G$ il n'y a aucun sommet de degré 1. Puis on a montré par le Lemme \ref{le:2pent} que si $v$ a un degré 2, alors il existe autour de $v$ au moins, deux pentagones ou un hexagone donc, $v$ peut distribuer par (R1) une unité de charge à chaque pentagone voisin ou deux unité à l'hexagone. De cette façon $v$ aura une nouvelle charge nulle et les faces autour restent positives vu que le même lemme prouve que tous les sommets autour de ces faces (sauf $v$) sont de degré 4. Par (R3) on sait que les sommets de degré 4 n'ont pas de charge à distribuer ni à recevoir.

%Pour un pair de sommets de degré 3, $v$ et $u$, on a trouvé grâce au Lemme \ref{le:33pent} que autour de $e = uv$ il y a au moins deux pentagones, ou une face de taille au moins 6, donc $u$ et $v$ peuvent distribuer sa charge vers ces faces grâce à (R2). Dans le cas où il y a deux pentagones, chacun des sommets distribuent sa charge vers un pentagone et sinon les deux sommets vers la même face. De cette manière $u$ et $v$ distribuent toute sa charge négative et la charge d'eux devient nulle et la charge totale de la configuration devient positive, car les faces n'ont que des sommets de degré 4 autour d'eux (sauf $v$ et $u$).

%Finalement, pour un sommet de degré 3, $v$ on a cherché par le Lemme \ref{le:3} au moins un pentagone autour qui décharge ce sommet, mais pour l'instant on a pas pu montrer cette réduction, donc c'est la seule situation qui nous reste pour prouver que le graphe se décharge complètement avec une somme de charge totale positive, ce qui concernerait une contradiction et montrerait la conjecture.

\section{Description de l'outil de réductibilité}
\label{chap:Des}

Ce chapitre sera consacré à introduire les méthodes et les notions utilisés pour le développement de l'outil que nous avons crée afin de vérifier la réductibilité des configurations. Suite aux besoins générées par les règles de déchargement définis dans le chapitre précédent, nous avons cherché à réduire des configurations déterminées.

%Dans ce chapitre on va présenter les méthodes qu'on a implémenté pour résoudre ce problème. D'abord, on va expliquer la méthode de réductibilité, un outil développé afin de automatiser la vérification de la réductibilité des configurations pertinentes. On verra après les configurations qu'on a défini qui pourraient nous empêcher de colorier la configuration ou le graphe avec 4 couleurs. Il faut remarquer que c'est nous qui ont défini les configurations et ses possibles réductions. Plus tard on montrera les différentes configurations et leur réductions, celles qui ont marché et quelques unes qui n'ont pas marché. 

%Depuis, on passera à l'aide des réductions, à définir les règles qui vont permet de décharger le graphe, et de cette façon de prouver que aucune configuration contredit la hypothèse de la coloration. On passe, donc au commentaire du code qu'on a conçu inspiré par le théorème de 4 couleurs.

\begin{definition}
Soit $H$ une configuration du graphe $G$ et soit $\partial_G(H)$ l'ensemble des arêtes sortantes tel que $|\partial_G(H)|=r$. Alors, une \emph{numérotation} de $H$ est l'identification des arêtes de $H$. D'abord les arêtes de cet ensemble s'identifient avec des chiffres  entre $1$ et $r$. Puis les arêtes restantes sont identifiées entre $r+1$ et $m$, étant $m$
\end{definition}

\begin{definition}
Soit $H$ une configuration du graphe $G$, $\partial_G(H)$ l'ensemble des arêtes sortantes et $\beta$ une coloration de $\partial_G(H)$. Alors un \emph{code de coloration} est un entier $k$ qui représente $\beta$ prenant les arêtes du ring et leur donnant une valeur selon leur couleur et leur position.

$$
k = \sum_{i=0}^{r-1} \c(\beta(i)) \cdot 4^{i} \qquad \textrm{tel que   } c(i) \in \{ 0,1,2,3 \} \qquad \textrm{pour  } i \in \{1,2,3,4 \}
$$
\label{de:code}
\end{definition}

\begin{definition}

\end{definition}
%\subsection{Réductibilité}

%La méthode de la réductibilité a été la méthode la plus travaillée, celle que nous a permis de supprimer beaucoup des configurations qui pourraient avoir créé un conflit. Pour ce fait on a développé un outil, qui lit des configurations et ses possibles réductions et elle vérifie si les possibles réductions le sont vraiment ou pas. Le code créé pour l'outil on l'a partitionné en 3 grandes parties (4 si on compte la façon de préparer les instances ou les configurations) qu'on passe à expliquer une par une ensuite.

%\subsection{Préparation des Configurations et leurs réductions}





%Cette première partie est pour expliquer comme est-ce qu'on défini les configurations pour pouvoir être lues. D'abord il  faut remarquer que on a défini les configurations par paires (configuration et sa possible réduction) où les deux configurations ont la même structure. Cette structure s'agit d'une première ligne avec juste le nom de la configuration (ou le numéro, pour reconnaître la structure), puis la prochaine ligne aura une chiffre qui détermine le nombre de sommets qui a la configuration, une deuxième chiffre concernant le nombre des arêtes extérieurs (ou sortants) et finalement une troisième chiffre qui sera un boolean avec un "0" si c'est la configuration et avec un "1" si c'est la réduction. Puis, à partir de la troisième ligne chaque ligne détermine le sommet courant, son degré, et les sommets voisins. Chaque configuration on la stockera dans une matrice d'adjacence.

%La façon de définir les configurations suit une ordre pour que les parties suivantes puissent être conçu de la façon qui nous intéresse, c'est à dire, les premiers sommets ce seront les sommets des arêtes extérieurs ou sortants, ceux qui vont connecter la configuration avec le reste du graphe. Ces sommets auront degré 1 et leurs arêtes incidentes conformeront l'ensemble des arêtes extérieurs ou \emph{ring}.



Une fois on a stocké le graphe dans une matrice on va chercher toutes les colorations des arêtes qui marchent et puis on va les stocker dans un vecteur sous forme de code. D'abord, avant de colorier le graphe, on va numéroter les arêtes. Pour ce fait on commencera par les arêtes extérieurs, et puis on numérotera les arêtes de l'intérieur par ordre croissant des sommets incidentes à chaque arête (on utilisera le sommet incidente le plus petit entre les deux).

....image avec exemple de numérotation...

Le prochain pas sera définir le voisinage de chaque arête, c'est à dire, stocker les arêtes adjacentes de chaque arête, mais pour ce qui nous intéresse on ne va stocker que les arêtes plus grandes (aven un numéro plus grande). La raison on l'expliquera au moment de colorier les arêtes. sachant que $\Delta \leq 4$ une arête elle ne pourrait pas avoir plus de 6 voisins.

... image avec exemple de voisinage

Finalement, la coloration, le pas principal pour lequel on a décidé de colorier la configuration en utilisant une méthode de \emph{backtracking}. Le \emph{backtracking} est une méthode qui revient en arrière sur des décisions déjà prises à chaque fois qu'on tombe sur une violation de contrainte ou que la solution courante ne admet aucune option. 

Dans notre cas, on partira sur la dernière arête, on la colorie et on parcours les arêtes de façon descendante et tant qu'on peut les colorier on continuera jusqu'à la première arête. Si on trouve un conflit, on revient sur la dernière arête coloriée et on utilise la prochaine couleur et on continue. Un conflit se crée quand une arête ne peut pas être coloriée car ses voisin sont déjà coloriées avec toutes les autres couleurs. Pour chaque arête ils existeront des couleurs "interdites" 

----exemple de graphe colorié----

Pour chaque bonne coloration, on ne gardera que la coloration des arêtes de ring car c'est celle-là la seule qui nous intéresse. Puis on stockera cette coloration sous forme du code multipliant $0,1,2$ ou $3$ (selon la couleur) par $4^{i}$, étant i l'arête coloriée.
\subsection{Génération de Couplages}

Cette partie est peut être la plus compliquée et la plus importante car c'est la partie dans laquelle on cherche des "mauvaises" colorations ou des colorations qui ne marchent pas normalement, mais qui peuvent se réduire vers des bonnes colorations, en utilisant des couplages entre les arêtes extérieurs ou sortants. 

D'abord on génère tous les couplages possibles à $k$ sommets extérieurs, on verra après qu'on aura besoin de tous. Puis on prendra les codes qui soient représentants des "mauvaises" colorations ou des colorations qui ne marchent pas. Le prochain pas ce sera de prouver si la coloration peut désormais bonne en faisant des échanges ou modifications. 

----image des chaînes de couleurs-----

La condition nécessaire et suffisante pour qu'une "mauvaise" coloration se réduise vers une que ce soit "bonne" est la suivante. S'il existe une paire de couleurs (qui colorient les arêtes extérieurs) pour lequel sur n'importe quelle couplage au nombre des arêtes colorés avec ces couleurs, on peut trouver une modification qui fait marcher la coloration, on peut dire que la "mauvaise" coloration se réduit vers une bonne (le code correspondant à la nouvelle coloration des arêtes extérieurs obtenue grâce aux échanges entre les arêtes coloriés à ces deux couleurs).

---image de mauvaise coloration qui se réduit vers une bonne et image de la bonne aussi-----

Il existe différentes types de "mauvaises" colorations, celles qui se réduisent vers une "bonne" coloration (on les appellera Type 1), celles qui se réduisent vers une "mauvaise" coloration du Type 1 ou de Type 2 (on appellera a ces dernières colorations "mauvaises de Type 2) et celles qui ne se réduisent pas. On fera donc, répéter ce processus de couplages en tant qu'il existe des colorations "mauvaises" qui se réduisent jusqu'à on ne trouve plus de colorations de Type 2.

\subsection{Colorations étendues}

Enfin, la dernière partie de la réductibilité, c'est la vérification. C'est à dire, on a déjà trouvé les colorations "bonnes" du graphe donné ainsi que les colorations de Type 1 et du Type 2 qui se réduisent vers les "bonnes" et qui desormaient "bonnes". Ensuite on cherchera toutes les colorations possibles (que les "bonnes") du graphe réduite et on regardera que toutes les colorations possibles du graphe réduite (la comparaison on la fait à partir des codes de coloration des arêtes sortants) correspondent aux "bonnes" colorations du graphe original. 

Autrement dit, si l'ensemble des "bonnes" colorations du graphe réduit est un sous ensemble des "bonnes" colorations du graphe original, alors le graphe est réductible et sa réduction est celle-ci qu'on a défini. Il faut toujours remarque qu'un graphe il a plusieurs possibles réductions. 
\subsection{Propriétés structurelles de $G$}

Ensuite nous montrons, grâce à l'outil de réductibilité, des configurations avec leurs réductions qui vont prouver les propriétés qu'on observe ci dessous. Ces propriétés vont permettre que les règles de déchargement définis dans le chapitre précédent entraînent une contradiction, c-à-d, nous allons montrer les propriétés qui assurent que le graphe se décharge de manière positive.

D'abord, regardons les propriétés qui se donnaient dans la conjecture précédente ($\Delta(G) \leq 4$, $g(G) \geq 5$, $\chi'(G) =4$) car dans ce cas là aussi elles sont présentes. Ces configurations des Lemmes \ref{le:14}-\ref{le:243}, nous les avons testé pour vérifier que le programme était correct. 

Tous les propriétés qu'on présente ensuite (les propriétés précédentes aussi) on les a prouvé à l'aide de l'outil de vérification des réductions qu'on a développé lors de ce {\color{blue}travail??}. Les configurations avec leurs réductions seront montrées dans le chapitre des résultats.

\begin{lemme}
Soit $G$ un contre-exemple minimal et $u$ et $v$ deux sommets voisins tels que $d_G(u)=d_G(v) = 3$. Alors parmi les 4 faces autour de ce paire de sommets il y a au moins deux pentagones, et tous les sommets autour des faces sauf $v$ et $u$ ont degré 4.
\label{le:33pent}
\end{lemme}

Pour prouver ce lemme nous avons vérifié que la configuration avec un pair de sommets $u$, $v$ avec 4 carrés autour de ce pair-ci était réductible (il existait déjà cette réduction). Puis nous avons trouvé une réduction pour ce pair de sommets avec 3 carrés et un pentagone autour, ce qui nous a permi prouver le lemme.

\begin{definition}
Pour un pair de sommets $u$ et $v$ de degré $3$ adjacentes, la distance à un autre sommet de degré $3$ est au moins $2$.
\label{de:3343}
\end{definition}

Pour prouver cette propriété nos avons vérifié une réduction pour une configuration $H$ "$3-3-4-3$" qui avait déjà été trouvée à la main. 


\begin{lemme}
Soit $G$ un contre-exemple et $v$ un sommet de $G$ tel que $d_G(v) = 2$. Alors, il existe autour de $v$ deux pentagones ou un hexagone et tous les sommets autour des faces sont de degré 4.
\label{le:2pent}
\end{lemme}

Dans ce cas là nous avons essayé d'utiliser la même procédure que pour le Lemme \ref{le:33pent}, c-à-dire, nous avons trouvé une réduction pour un sommet $v$ de degré 2 avec deux carrés autour de lui, et pareil pour $v$ avec un carré et un pentagone. Malheureusement nous n'avons pas encore trouvé cette dernière réduction.

\begin{lemme}
Soit $G$ un contre-exemple minimal et $v \in V(G)$ tel que $d_G(v) = 3$.
\label{le:3}
\end{lemme}

Dans ce cas-là non plus, nous n'avons  n'avons pas réussi à réduire le cas où le sommet $v$ a 3 carrés autour de lui, mais on croit forcement en la vérité de ce lemme.

\section{Résultats}
\label{chap:res}

Dans ce dernier chapitre, nous montrons les résultats obtenus grâce à l'outil de réductibilité. D'abord nous avons vérifié sur des configurations réductibles déjà connues qui montrent les Lemmes \ref{le:23}-\ref{le:333}, puis nous avons testé l'outil pour des configurations plus complexes qui ont été déjà prouvées aussi (Lemme \ref{le:33pent} (premier cas) et Lemme \ref{de:3343}) et finalement nous présentons les nouvelles configurations trouvées ainsi que ses réductions correspondantes lesquelles ont aidé a prouver certaines propriétés structurelles (Lemme \ref{le:33pent} (deuxième cas) et Lemme \ref{le:2pent})  

\subsection{Vérification des configurations réductibles connues}
\subsection{Pair de sommets de degré 3}
\subsection{Sommet de degré 2}
\subsection{Sommet de degré 3}

\section{Conclusion}
\label{chap:concl}
Finalement nous allons faire quelques remarques autour du sujet en soi ainsi que autour du stage de recherche et leurs apports lors de ces 5 mois de stage.

Tout d'abord, un peu avant le démarrage du stage j'ai travaillé sur les notions de base de la réductibilité et le déchargement comme une mise a niveau. Pour ce fait j'ai montré à la main les deux conjecturés que nous avons présenté dans le chapitre \ref{chap:easy} afin de maîtriser ces deux techniques. Au même temps, une lecture exhaustive des articles du Théorème de 4 couleurs {\color{blue}citation} et de la conjecture des graphes planaires de classe 1 pour $\Delta = 7$ \cite{SandersZhao} m'a permis de bien comprendre le coté technique de la réductibilité. 

Une fois les aspects techniques et pratiques ont été compris, j'ai récupéré le code du programme qui traite la réductibilité du Théorème de 4 couleurs car nous nous sommes inspirés de ce programme pour développer le nôtre. Tout ce premier mois j'ai travaillé sur la compréhension du code, afin d'identifier les parties que nous pouvions utiliser, les parties à modifier et les parties à refaire. Bien que nous nous sommes inspirés de ce programme, nous n'avons utilisé que la fonction de lecture et la génération de colorations, lesquelles nous avons dû modifier. J'ai vu que l'inexistence de parité ainsi que des cas manquantes qui seraient présentes dans notre problème pouvaient créer des difficultés donc nous avons décidé refaire tout le reste du programme.

J'ai utilisé la plupart de temps à développer l'outil et à la recherche des nouvelles configurations qui montrent les propriétés structurelles qui pouvaient nous intéresser. Finalement, le dernier mois et demi du stage je l'ai consacré à la technique de déchargement et a la recherche des deux derniers cas que nous n'avons pas été capables de trouver.

Pourtant, malgré ces deux derniers cas que nous n'avons pas trouvé je considère que nous avons contribue à la démonstration d'une conjecture qui n'a pas été prouve depuis qu'elle a été conçu et nous avons trouvé des nouvelles configurations qui montrent des propriétés structurelles cruciales. Je crois que c'est possible trouver ces deux dernières réductions et même si le stage fini ici, nous allons continuer jusqu'au but. 

Au niveau personnel, ce stage a été un défi, une expérience très enrichissante qui m'a permis d'une coté améliorer les compétences acquises lors de ma formation,et d'une autre coté apprendre des nouvelles techniques ainsi que des outils nécessaires pour le travail de recherche. Je dois remarquer la présence constante et le soutien de mon encadrant Monsieur Kardo\v s, qui m'a appris des aspects surtout liés à ce qui concerne être un chercheur et qui m'a aidé durant l'écriture de l'article, vu qu'il s'agit d'un article scientifique qui demande de la neutralité et de l'objectivité. Ce a été, a mon avis au même temps que la compréhension des aspects techniques les parties qui m'ont semblé les plus difficiles.

Pour conclure, ce stage m'a apporté la confiance, la force et l'envie de continuer ma formation avec une thèse. Bien que ce soit vrai que je hésitait à continuer, ce stage m'a donné une piste du type de travail que je souhaiterait effectuer dans les prochains ans.



\begin{thebibliography}{99}
\bibitem{ABH} R.~E.~L.~Aldred, S.~Bau, D.~A.~Holton, and B.~D.~McKay,	
Nonhamiltonian 3-Connected Cubic Planar Graphs,
SIAM J. Discrete Math. 13 (1) (2000), 25--32.
\bibitem{br} G.~Brinkmann and A.~W.~M.~Dress, A Constructive Enumeration of Fullerenes, J. Algorithm. 23 (2) (1997), 345--358.
\bibitem{erman} R.~Erman, F.~Kardo\v s, and J. Mi\v skuf, Long cycles in fullerene graphs, J. Math. Chem. 46 (4) (2009), 1103--1111.
\bibitem{good} P.~R.~Goodey, A class of hamiltonian polytopes, (special issue dedicated to Paul Tur\'{a}n) J. Graph Theory 1 (1977) 181--185.
\bibitem{good2} P.~R.~Goodey, Hamiltonian circuits in polytopes with even sided faces, Israel J. Math. 22 (1975) 52--56.
\bibitem{JO} S.~Jendrol$\!$'~and P.~J.~Owens, {Longest cycles in generalized Buckminsterfullerene graphs}, J. Math. Chem. {18} (1995) 83--90.
\bibitem{kral} D.~Kr\'al$\!$', O.~Pangr\'ac, J.-S.~Sereni, and R.~\v Skrekovski, Long cycles in fullerene graphs, J. Math. Chem. 45 (4) (2009), 1021--1031.



\bibitem{NP} M.R. Garey, D.S. Johnson, L. Stockmeyer, Some simplified NP-complete problems, Proceedings of the Sixth Annual ACM Symposium on Theory of Computing (1974) 47--63.
\bibitem{Vizing} V.G.~Vizing, On an estimate of the chromatic class of a p-graph, Diskret. Analiz. (3) (1964), 25--30.
\bibitem{Vizing2} V.G.~Vizing, Critical graphs with given chromatic class, Metody Diskret. Analiz (5) (1965), 9--17.
\bibitem{SandersZhao} D.P. Sanders et Y. Zhao, Planar graphs of maximum degree seven are class I, J. Combin. Theory Ser. B, {83} (2), (2001), 201--212.
\bibitem{A&H} K. Appel, W. Haken, Every Planar Map is Four Colorable. I. Discharging, Illinois Journal of Mathematics 21 (3), (1977), 429--490
\bibitem{T4C} N.Robertson, D.P.Sanders, P.Seymour, R.Thomas, The Four-Colour Theorem, J. Combin. Theory Ser. B 70 (1), (1997)  2--44,
\end{thebibliography}




\end{document}