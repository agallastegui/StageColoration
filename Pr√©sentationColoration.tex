\documentclass{beamer}
\usepackage{mathrsfs}
\usepackage[utf8]{inputenc}
\usepackage[french]{babel}
\usepackage{array,multirow,makecell}
\setcellgapes{1pt}
\makegapedcells
\usepackage[T1]{fontenc}
\usepackage{graphicx}
\mode<presentation>
{ \usetheme{Berlin}
%  \setbeamercovered{transparent}
}
\usepackage{times}
\usepackage[T1]{fontenc}

\newtheorem{lemme}{Lemme}
\newtheorem{proposition}{Proposition}

\newtheorem{conjecture}{Conjecture}

\def\ceps{3656}

\title[Coloration des arêtes des graphes planaires]{Coloration des arêtes des graphes planaires}
\author[A. Gallastegui and F. Kardo\v s]{\underbar{Antonio Gallastegui} \and Franti\v sek Kardo\v s}
\institute[]{
Laboratoire Bordelais de Recherche Informatique \\
Université de Bordeaux\\
}
\date{Bordeaux, 10 juin, 2016}


% If you wish to uncover everything in a step-wise fashion, uncomment
% the following command: 

%\beamerdefaultoverlayspecification{<+->}


\begin{document}

\begin{frame}
  \titlepage
\end{frame}
%\begin{frame}
% \tableofcontents
%\end{frame}
\section{Introduction}
\subsection{Notions de Base}
\begin{frame}
Le degré $d_G(v)$ de $v\in V(G)$ est le nombre de sommets voisins de $v$.
\begin{itemize}
\item $\Delta(G)$ degré maximum
\item $\delta(G)$ degré minimum
\end{itemize} 
\begin{figure}[ht]
\centerline{
\includegraphics[scale=1]{./Figures/Fig-def.1}
}
\caption{Un exemple d'un graphe de degré minimum 2 et degré maximum 4}
\end{figure}


La \emph{maille} $g(G)$ d'un graphe $G$ est la longueur d'un plus petit cycle dans $G$, s'il en existe un. Sinon $g(G)$ est infinie.
\end{frame}

\begin{frame}
Soit un graphe $G$. Un \emph{couplage} $M$ est un ensemble d'arêtes deux à deux non-adjacentes.
\begin{figure}[ht]
\centerline{
\includegraphics[scale=1]{./Figures/Couplage.1}
\hfil
\includegraphics[scale=1]{./Figures/Couplage.2}
}
\caption{Un exemple d'un couplage maximal (gauche) et un couplage maximum que dans ce cas là c'est aussi un couplage parfait (droite)}

\end{figure}

\end{frame}

\begin{frame}
Le graphe $G$ est \emph{planaire} ssi il peut être dessiné sur le plan sans croisement des arêtes.
Une face $f$, est une région connexe délimitée par des arêtes. Il y a toujours une face infinie.
\pause
\begin{theorem}[Euler, 1750]
%\underbar{Formule d'Euler (1750)}
Soit $G$ un graphe planaire connexe à $n$ sommets, $m$ arêtes et $f$ faces. Alors,
$$
n-m+f=2
$$
\end{theorem}
\begin{figure}[ht]
\centerline{
\includegraphics[scale=1]{./Figures/Fig-def.2}
}
\end{figure}
\end{frame}


\subsection{Coloration de Graphes}
\begin{frame}
Soit $G$ graphe, une \emph{coloration propre} de $G$ est application $\varphi: V(G) \to C$ où $C=\{1,2,...,k\}$ ensemble d'entiers, t.q. $\forall uv\in E(G) \implies\varphi(u) \neq \varphi(v)$. 

Le nombre chromatique $\chi(G)$ est le nombre de couleurs minimum pour colorier $G$.

\begin{figure}[ht]
\centerline{
\includegraphics[scale=1]{./Figures/Coloration.1}
}
\caption{Un exemple d'une coloration propre d'un graphe $G$ avec $\chi(G)=3$.}
\end{figure}

\end{frame}
\begin{frame}
Soit $G$ un graphe. Une coloration d'arêtes de $G$ est une application $\varphi:E(G) \to C$, où $C=\{1,2,...,k\}$ t.q. $\forall e$, $e' \in E(G)$ adjacentes $\implies$ $\varphi(e) \neq \varphi(e')$. 

L'indice chromatique $\chi'(G)$ est le nombre de couleurs minimum qu'on nécessite pour colorier les arêtes de $G$.

\begin{figure}[ht]
\centerline{
\includegraphics[scale=1]{./Figures/Coloration.2}
}
\caption{Un exemple d'une coloration d'arêtes d'un graphe $G$ avec $\chi'(G)=3$.}
\end{figure}
\end{frame}

\begin{frame}
\begin{theorem}[Vizing]
Soit $G$ un graphe simple. Alors $\Delta(G) \leq \chi'(G) \leq \Delta(G) + 1$.
\end{theorem}
Si $\chi'(G) = \Delta(G)$,  $G$ est classe 1.

\pause
%Pour un graphe planaire $G$ voici quelques résultats connus
Quels graphes planaires sont classe 1?
\begin{itemize}
\item $\Delta(G) \geq 8$ $\implies$ $\chi'(G)=\Delta$. (Vizing 1965).
\item $\Delta(G) = 7$ $\implies$ $\chi'(G)=7$. (Sanders, Zhao 2001).
\item $\Delta(G) = 5$ $g(G) \geq 4$ $\implies$ $\chi'(G) = 5$.
\item $\Delta(G) = 4$ $g(G) \geq 5$ $\implies$ $\chi'(G) = 4$.
\end{itemize}
\begin{conjecture}
Soit $G$ un graphe planaire. Alors,

$\Delta(G) = 4$ $g(G) \geq 4$ $\implies$ $\chi'(G) = 4$
\end{conjecture}
\end{frame}

\begin{frame}
\begin{center}
\begin{tabular}{|  c  |  c  |  c  |  c  |  c  |  c  |  c  |}
\hline $\Delta$ & 3 & 4 & 5 & 6 &  7  &  8  \\
\hline $g \geq 3$ & {\color{red} \textsf{X}} & {\color{red} \textsf{X}}  & {\color{red} \textsf{X}} & {\color{blue} \textsf{?}} &  {\color{green}\checkmark} &  {\color{green}\checkmark}\\
\hline $g \geq 4$ & {\color{red} \textsf{X}} & {\color{blue} \textsf{?}}  &  {\color{green} \checkmark} &  &  &   \\
\hline $g \geq 5$ & {\color{red} \textsf{X}} & {\color{green}\checkmark} &  &  &  &  \\
\hline
\end{tabular}
\end{center}
\end{frame}

\section{Résultats Connus}
%\begin{frame}
%Soit $G$ graphe avec arêtes coloriées et $G[c1,c2]$ sous graphe de $G$ induit par arêtes coloriées $c1$ et $c2$. 
%
%Tout composante connexe $C$ de $G[c1,c2]$ est une chaîne à laquelle renversants les couleurs on obtient nouvelle coloration de $G$.
%
%Pour $e\in E(G)$ t.q. $\varphi(e)=c$ et une autre couleur $c'$. Alors il existe une et une seule chaîne de $G[c,c']$ contenant $e$. C'est \emph{$(c, c')$-chaîne de Kempe}. 
%\end{frame}

\subsection{$\Delta(G)=5$, $g(G)\geq 4$ $\implies$ $\chi'(G)=5$}
\begin{frame}
\begin{proposition}
Soit $G$ un graphe planaire avec $\Delta(G) = 5$ et $g(G) \geq 4$. Alors $\chi'(G)=5$.
\end{proposition}

\end{frame}
\begin{frame}
Propriétés structurelles de un contre-exemple minimal $G$:
\begin{itemize}
\item Il n'existe pas de sommet $v$ de $G$ tel que $d_G(v) = 1$.
\item Soit $v \in V(G)$ tel que $d_G(v) = 2$. Alors $\forall u \in N(v)$, $d_G(u) = 5$.
\item Soit $v\in V(G)$ tel que $d_G(v) = 3$. Alors, $\forall u \in N(v)$, $4\le d_G(u) \le 5$.
\item Soit $v\in V(G)$ tel que $d_G(v) = 3$. Alors, au moins deux voisins de $v$ sont de degré $5$.
\item Soit $uv \in E(G)$ tel que $d_G(u) = 2$ et $d_G(v) = 5$. Alors, tous les voisins de $v$ sauf $u$ sont de degré $5$. 
\item Soit $v \in V(G)$ tel que $d_G(v) = 5$. Alors, $v$ a au plus deux voisins de degré $3$.
\end{itemize}
\end{frame}

\begin{frame}
Déchargement:

La charge initiale est définie par
 $$
\gathered
w(v) = d_G(v) - 4 \qquad \textrm{pour tout $v\in V(G)$,} \\
w(f) = d_G(f) - 4 \qquad \textrm{pour tout $f\in F(G)$,} 
\endgathered 
$$

$$ 
\gathered
w(v) = -2 \qquad \textrm{pour tout $v \in V(G)$, tel que } d_G(v) = 2, \\
w(v) = -1 \qquad \textrm{pour tout $v \in V(G)$, tel que } d_G(v) = 3, \\
w(v) =  0 \qquad \textrm{pour tout $v \in V(G)$, tel que } d_G(v) = 4, \\
w(v) =  1 \qquad \textrm{pour tout $v \in V(G)$, tel que } d_G(v) = 5.
\endgathered
$$

La charge initiale de toutes les faces est positive puisque $g(G)\ge 4$.
\end{frame}

\begin{frame}
\begin{lemme}
Soit $G$ un graphe planaire connexe. Alors,
$$ \sum_{v \in V(G)} (d_G(v) - 4) + \sum_{f\in F(G)} (d_G(f) -4) = -8.$$
\end{lemme}
Preuve.
$$
\gathered 
\sum_{v \in V(G)} (d_G(v) - 4) + \sum_{f\in F(G)} (d_G(f) -4) = 2m -4n + 2m - 4f = \\
 =-4(n - m + f) = -4\cdot 2 = -8.
\endgathered
$$
\end{frame}

\begin{frame}
Les règles de déchargement:
\begin{enumerate}
\item[(R1)] Tout sommet de degré 5 donne une unité de charge à chacun de ses voisins de degré 2.
\item[(R2)] Tout sommet de degré 5 donne $1/2$ de charge à chaque voisin de degré 3.
\end{enumerate}
\end{frame}

\subsection{$\Delta(G)=4$, $g(G)\geq 5$ $\implies$ $\chi'(G)=4$}
\begin{frame}
\begin{proposition}
Soit $G$ un graphe planaire avec $\Delta(G) = 4$ et $g(G) \geq 5$. Alors $G$ est $4$-arête-coloriable.
\end{proposition}
Propriétés structurelles d'un contre-exemple minimal $G$:
\begin{itemize}
\item Il n'existe pas de sommet $v$ de $G$ tel que $d_G(v) = 1$.
\item Soit $v \in V(G)$ tel que $d_G(v) = 2$. Alors $\forall u \in N(v)$, $d_G(u) = 4$.
\item Soit $uv \in E(G)$ tel que $d_G(u) = 2$ et $d_G(v) = 4$. Alors, tous les voisins de $v$ sauf $u$ sont de degré $4$.
\item Soit $v \in V(G)$ tel que $d_G(v)=3$. Alors, $v$ a au moins deux voisins de degré $4$.
\end{itemize}
\end{frame}

\begin{frame}
Déchargement:

La charge initiale est définie par,

$$
\gathered
w(v) = \frac{3}{2} d_G(v) - 5 \qquad \textrm{pour tout $v\in V(G)$,} \\
w(f) = d_G(f) - 5 \qquad \textrm{pour tout $f\in F(G)$.}
\endgathered
$$

\pause

$$
\gathered
w(v) = -2 \qquad \textrm{pour tout $v \in V(G)$, tel que } d_G(v) = 2,\\
w(v) = -\frac{1}{2} \qquad \textrm{pour tout $v \in V(G)$, tel que } d_G(v) = 3,\\
w(v) =  1 \qquad \textrm{pour tout $v \in V(G)$, tel que } d_G(v) = 4.
\endgathered
$$

D'ailleurs, la charge initiale de toutes les faces est positive puisque $g(G) \geq 5$.
\end{frame}

\begin{frame}
\begin{lemme}
Soit $G$ un graphe planaire connexe. Alors,
$$
\sum_{v \in V(G)} \left(\frac{3}{2} d_G(v) - 5\right) + \sum_{f\in F(G)} \left(d_G(f) -5\right) = -10.
$$
\end{lemme}
Preuve.
$$ 
\gathered
\sum_{v \in V(G)} \left(\frac{3}{2}d_G(v) - 5\right) + \sum_{f\in F(G)} \left(d_G(f) -5\right) =\\
= 3m -5n + 2m - 5f = -5(n - m + f) = -5 \cdot 2 = -10.
\endgathered
$$
\end{frame}

\begin{frame}
Les règles de déchargement:
\begin{enumerate}
\item[(R1)] tout sommet de degré 4 donne $1$ de charge à chacun de ses voisins de degré 2.
\item[(R2)] tout sommet de degré 4 donne $1/4$ de charge à chaque voisin de degré 3.
\end{enumerate}
Dans ce cas là aussi, la somme totale de la charge reste invariante et la charge de toutes les faces reste positive.
\end{frame}


\section{Réductibilité}
\subsection{ }
%\begin{frame}
%Les démonstrations de Lemmes \ref{le:1}-\ref{le:5333} et \ref{le:14}-\ref{le:333} ont toutes la même structure: La supposition que le contre-exemple minimal $G$ contienne un arrangement de sommets de certains degrés entraîne une contradiction avec non-colorabilité de $G$.
%  
%
%Nous allons généraliser cette notion.
%\end{frame}

\begin{frame}
Soit $H$ un graphe étiqueté avec une fonction $\gamma$: $V(G) \to \mathbb{N}$ telle que $\forall v \in V(H)$ on a $\gamma(v) \geq d_H(v)$. Le graphe $H$ est une \emph{configuration} dans $G$ s'il existe un sous-graphe $X$ de $G$ et un isomorphisme $\xi$: $H \to X$ tel que $\forall v \in V(H)$, $\gamma(v) = d_G(\xi(v))$. 

\pause


Dans la suite, lorsque nous parlons d'une configuration $H$ dans un graphe $G$, nous identifions $H$ et sa copie $X=\xi(H)$ dans $G$.
\end{frame}

\begin{frame}
Soit $H$ une configuration de $G$. Notons
$$
\partial_G(H) = \{e =uv \in E(G) : u \in V(H)\textrm{ et }v \in V(G \setminus H)\}.
$$ 
%l'ensemble des arêtes sortantes de $H$ et $r=|\partial_G(H)|$.

\pause


Notons $H^* = H \cup \partial_G(H)$ et $G'=G\setminus H \cup \partial_G(H)$.
\end{frame}

%\begin{frame}
%Si $\varphi:E(G)\to \{1,2,3,4\}$ est une 4-arête-coloration de $G$, alors $\varphi$ induit une coloration de $\partial_G(H)$, de $H^*$, et de $G'$. Pour capturer la structure de la coloration $\varphi$ dans $G'$, nous introduisons les notions et notations suivantes :
%\end{frame}

\begin{frame}
Soit 
\begin{itemize}
\item $G$ un graphe avec $\Delta(G)=4$ et $H$ une configuration de $G$,
\item $\varphi:E(G')\to \{1,2,3,4\}$ une 4-arête-coloration de $G'$,
\item $c, c' \in \{1,2,3,4\}$, $c\ne c'$ deux couleurs quelconques. 
\end{itemize}

Une coloration $\varphi':E(G)\setminus E(H)\to \{1,2,3,4\}$ est \emph{adjacente} à $\varphi$, si $\varphi'$ peut être obtenu à partir de $\varphi$ en alternant les couleurs d'une $(c,c')$-chaîne de Kempe de $e$ pour un choix d'une arête $e\in\partial_G(H)$ et des couleurs $c,c'\in\{1,2,3,4\}$. 



Notons $N(\varphi)$ l'ensemble des colorations adjacentes à $\varphi$.
\end{frame}

\begin{frame}
Soit 
\begin{itemize}
\item $G$ un graphe avec $\Delta(G)=4$ et $H$ une configuration de $G$,
\item $\varphi:E(G')\to \{1,2,3,4\}$ une 4-arête-coloration de $G'$,
\item $c, c' \in \{1,2,3,4\}$, $c\ne c'$ deux couleurs quelconques. 
\end{itemize}



Le \emph{graphe résiduel}
$\mathcal{X}^{\varphi}_{c,c'}$ est défini par
\begin{itemize}
\item $V(\mathcal{X}^{\varphi}_{c,c'}) = \partial_G(H)$, 
\item $E(\mathcal{X}^{\varphi}_{c,c'}) = \{ e_ie_j$ tels que $(c,c')$-chaîne de Kempe de $e_i$ est la même que la $(c,c')$-chaîne de Kempe de $e_j\}$.
\end{itemize}

\pause

Le graphe résiduel est donc un couplage planaire de $\partial_G(H)$.
\end{frame}

\begin{frame}
\begin{proposition}
Soit 
\begin{itemize}
\item $G$ avec $\Delta(G)=4$ et $H$ une configuration de $G$,
\item $\varphi:E(G')\to \{1,2,3,4\}$ 4-arête-coloration de $G'$, 
\item $\varphi' \in N(\varphi)$ une coloration adjacente à $\varphi$,
\item $\beta = \varphi|_{\partial_G(H)}$ et $\beta' = \varphi'|_{\partial_G(H)}$.
\end{itemize}
Alors,
$$
|\{e\in \partial_H(G) : \beta(e)\ne \beta'(e)\}| \in \{1,2\},
$$
de plus, on a \\
$|\{e\in \partial_H(G) : \beta(e)\ne \beta'(e)\}|=1$ $\Leftrightarrow$ $d_{\mathcal{X}_{\varphi}^{c,c'}}(e)=0$, et \\
$|\{e\in \partial_H(G) : \beta(e)\ne \beta'(e)\}|=2$ $\Leftrightarrow$ $d_{\mathcal{X}_{\varphi}^{c,c'}}(e)=1$. 
\end{proposition}
\end{frame}

\begin{frame}
Considérons des coloration de $H$.
\end{frame}

\begin{frame}
Soit $H$ une configuration de $G$ et $\beta$ une coloration de $\partial_G(H)$. 

Alors, $\beta$ \emph{s'étend} vers $H$ si $\exists \psi$ de $H^*$  t.q.   $\psi|_{\partial_G(H)}= \beta$

$$
\Phi_0 = \{\beta : \exists \psi : E(H^*)\to\{1,2,3,4\} \textrm{t.q.} \psi|_{\partial_G(H)} = \beta\}.
$$
Pour $i\in\mathbb{N}$, soit
$$
\gathered
\Phi_{i+1} = 
\{ 
\beta : 
\beta\in \Phi_i
\textrm{ ou }
\exists c,c'\in \{1,2,3,4\},\,\\ 
\forall X\textrm{ couplage planaire de $\partial_G(H)$, }
N^X_{c,c'}(\beta) \cap \Phi_i \neq \emptyset 
\},\\
\Phi(H)=\bigcup_{i\in \mathbb{N}} \Phi_i(H).
\endgathered
$$
\end{frame}

\begin{frame}
\begin{theorem}
Soit 
\begin{itemize}
\item $H$ une configuration de $G$. 
\item $H'$ une autre configuration t.q. $\partial(H)=\partial(H')$.
\item $G'$ le graphe obtenu à partir de $G$ en remplaçant $H$ par $H'$.
\end{itemize}

Si $\Phi_0(H')\subseteq \Phi(H)$, alors, pour toute 4-arête-coloration $\varphi'$ de $G'$ il existe une 4-arête-coloration $\varphi$ de $G$.
\end{theorem}

Preuve.
il suffit de modifier la coloration $\varphi'$ le long des chaînes de Kempe un nombre fini de fois afin d'obtenir une coloration qui s'étend directement vers $H$.
\end{frame}

\begin{frame}
Soit $H$ et $H'$ deux configurations telles que $\partial(H)=\partial(H')$. La configuration $H$ \emph{se réduit} vers $H'$ si $\Phi_0(H')\subseteq \Phi(H)$. La configuration $H$ est \emph{réductible} s'il existe une configuration $H'$ vers la quelle elle se réduit.

%\pause
%
%Ce théorème permet d'automatiser la vérification de réductibilité de configurations -- on n'a pas besoin de connaître la structure du graphe $G$, il suffit de deviner la réduction (la configuration $H'$) et vérifier que la condition de la définition dé réductibilité est satisfaite.
\end{frame}

\section{Description de l'outil}

\begin{frame}
C'est un programme qui prend,
\begin{itemize}
\item Entrée:  Une configuration $H$ et sa réduction potentielle $H'$.
\item Sortie:  Une réponse de OUI/NON à la question de réductibilité.
\end{itemize}
Ensuite, nous décrivons l'outil de la manière suivante,
\begin{itemize}
\item Pré-calcul,
\item Traitement,
\item Implémentation.
\end{itemize} 
\end{frame}

\subsection{Pré-calcul}
\begin{frame}
\begin{itemize}
\item Génération de Couplages de $r$ sommets, pour $r$ allant de 1 à MAXRING.
\item Variation entre deux codes adjacents (Plus Tard).
\end{itemize}
\end{frame}

\subsection{Traitement}

\begin{frame}
Partant d'une configuration $H$ de $G$ et $H'$ sa réduction potentielle,
\begin{itemize}
\item On génère l'ensemble des colorations $\psi$ de $H$ qui constituent $\Phi_0$,
\item Tant que $\Phi_i \subset \Phi_{i+1}$, on génère des colorations adjacentes à $\beta \in \Phi_i$,
\item L'ensemble totale des colorations qui s'étendent vers $H$ est $\Phi(H) = \bigcup_{i \in \mathbb{N}} \Phi_i(H)$,
\item On génère l'ensemble $\Phi'_0$ des colorations $\psi'$ de $H'$ t.q. une coloration de $G'$ s'étend vers $H'$ et
\item On regarde si $\Phi'_0(H') \subseteq \Phi(H)$.
\end{itemize}
\end{frame}

\begin{frame}
Alors,
pour $\beta'$ quelconque qui ne s'étend pas,
s'il existe une paire de couleurs $c$, $c'$ pour laquelle n'importe quel couplage des arêtes coloriées avec $c$ ou $c'$ il existe un code $\beta$ adjacent à $\beta'$, $\beta'$ s'étend.
\end{frame}

\subsection{Implémentation}

\begin{frame}
\underbar{Lecture de $H$ et $H'$:}
\begin{itemize}
\item Lecture de configuration $H$ et sa réduction potentielle $H'$ présentées sous forme d'une liste d'adjacence.
\item Les deux seront stockées dans deux matrices différentes.
\end{itemize}
\end{frame}

\begin{frame}
\underbar{Coloration de $H$:}

Nous commençons par numéroter les arêtes du graphe $H$, commençant par les arêtes sortantes et nous continuons vers l'intérieur.

Nous colorions les arêtes de $H$ partant de la dernière arête vers l'extérieur à l'aide de la méthode de retour sur le trace (Backtrack).

Nous n'allons stocker que la coloration $\beta=\psi|_{\partial_G(H)}$ sous forme de code suivant,
$$
k = \sum_{i=0}^{r-1} c(\beta(e)) \cdot 4^{e} \qquad \textrm{tel que   } c(i) \in \{ 0,1,2,3 \} \qquad \textrm{pour  } i \in \{1,2,3,4 \}.
$$

$\Phi_0(H)$ est un ensemble de $k$.
\end{frame}

\begin{frame}
\underbar{Colorations voisines:}

Soit $H$ une configuration, une coloration $\beta$ avec son code $k$ de $\partial_G(H)$ et deux couleurs $c$ et $c'$. 

Alors $\mathcal{M}$ est l'ensemble de tous les couplages $M$ possibles des arêtes de $\partial_G(H)$ coloriées avec $c$ ou $c'$.

\pause

Alors, il existe des colorations $\beta'$ de $\partial_G(H)$ avec code de coloration $k'$ \emph{adjacentes} de $\beta$ si pour une paire de couleurs $c$ et $c'$,
$$
\begin{cases}
k' =  k - (2\beta'(e)+c + c')\cdot 4^{e} \textrm{ si  }d_M(e)=0 \\
k' = k  - (2\beta'(e)+c + c')\cdot 4^{e} - (2\beta'(e')+c + c')\cdot 4^{e'} \textrm{ si  }d_M(e)=1.
\end{cases}
$$

Cet ensemble de codes de coloration $k'$ forment $\Phi_{i+1}$ pour tout $i\in \mathbb{N}$.
\end{frame}


%Soit $H$ configuration de $G$ et $\partial_G(H)$ t.q. $|\partial_G(H)|=r$. 
%
%Alors, une \emph{numérotation} de $H$ est l'identification des arêtes de $H$.
%
%\pause
%
%Soit $H$ configuration de $G$ numérotée. 
%
%Alors, le \emph{voisinage} de $e$ est 
%$$
%N(e) = \{e': \textrm{ numérotation de $e'$ est plus grande}\}
%$$
%\end{frame}
%
%\begin{frame}
%Soit $H$ configuration de $G$ numérotée. 
%
%Alors, une \emph{coloration des arêtes} $\psi$ de $H$ est obtenu partant du sommet $v$ numéroté avec $m$ et coloriant un par un, sans créer conflits de voisinage, à l'aide du méthode de retour sur le trace (Backtrack).
%
%\pause
%
%-image-
%\end{frame}
%
%\begin{frame}
%Ces colorations correspondent aux colorations de l'ensemble $\Psi_0$ de coloration qui s'étendent.
%
%\pause
%
%Pour colorier les arêtes de $H$, nous utilisons les couleurs $1$, $2$, $4$ et $8$, sous forme binaire. Pour un nombre $r$ d'arêtes sortantes de $H$ il existe $4^r$ possibles colorations. Nous allons s'intéresser aux colorations des arêtes sortantes de $H$, c-à-d deux colorations différentes de $H$ avec une même coloration des arêtes sortantes sont prises comme une même coloration.
%\end{frame}
%
%\begin{frame}
%Soit $H$ configuration du graphe $G$, $\partial_G(H)$  et $\beta$ coloration de $\partial_G(H)$. 
%
%Alors un \emph{code de coloration} est un entier $k$ qui représente $\beta$.
%
%
%\end{frame}

%\begin{frame}
%Soit $H$ configuration de $G$, une coloration $\beta$ de $\partial_G(H)$ et deux couleurs $c$ et $c'$. 
%
%Alors $M$ est l'ensemble de tous les couplages possibles des arêtes de $\partial_G(H)$ coloriées avec $c$ ou $c'$. 
%
%\pause
%
%Soit 
%\begin{itemize}
%\item $H$ configuration de $G$\\
%\item $\Psi_0$ l'ensemble de $\beta$ colorations étendues vers $H$ \\
%\item $k$ Codes de colorations des $\beta$. 
%\end{itemize}
%Alors, il existe des colorations $\beta'$ de $\partial_G(H)$ avec code de coloration $k'$ \emph{adjacentes} de $\beta$ si pour une paire de couleurs $c$ et $c'$,
%$$
%\begin{cases}
%k' =  k - (2\beta'(e)+c + c')\cdot 4^{e} \textrm{ si $e \in X$ renverse ses couleurs}\\
%k' = k  - (2\beta'(e)+c + c')\cdot 4^{e} - (2\beta'(e')+c + c')\cdot 4^{e'} \textrm{ si $(e,e')$-chaîne de Kempe renverse ses couleurs}
%\end{cases}
%$$
%\end{frame}

%\begin{frame}
%Soit $H$ configuration de $G$ et $\Psi_0$.
%
% Alors, pour tout $i \in \mathbb{N}$ $\Psi_{i+1}$ est l'ensemble des colorations $\beta'$ de $\partial_G(H)$ adjacentes aux colorations $\beta$ de $\Psi_i$.
% 
%\pause
%
%$$
%\Psi = \Psi_0 \bigcup_{i\in \mathbb{N}}\Psi_i.
%$$
%\end{frame}

%\begin{frame}
%\begin{theorem}
%Soit $H$ configuration de $G$ et $H'$ une configuration t.q. $\partial_G(H)=\partial_G(H')$. 
%
%Alors,
%\begin{center}
%$H'$ est une réduction de  $H$ $\Leftrightarrow$ $\Psi_0$  $\Psi'_0 \subseteq \Psi$.
%\end{center}
%\end{theorem}
%\end{frame}

\section{Déchargement}
\subsection{ }
\begin{frame}
Soit $G$ un graphe planaire. V(G) ensemble de sommets et F ensemble de faces de $G$.

La \emph{charge initiale} des sommets et des faces,

$$
\gathered
w: V(G)\cup F(G) \to \mathbb{R}, \\
w(x) = d_G(x) - 4. \\
\endgathered
$$
\end{frame}

\begin{frame}
\begin{lemme}
Soit $G$ un graphe planaire et $w$ la application de la charge initiale. Alors la charge initiale totale du graphe est $-8$.
\end{lemme}

Preuve.
$$
\gathered
\sum_{u \in V(G)} (d_G(v) - 4) + \sum_{f \in F} (d_G(f) - 4) = 2m - 4n + 2m - 4f =\\
= -4( n - m + f) = -4 \cdot 2 = -8.
\endgathered
$$
\end{frame}

\begin{frame}
Alors les charges pour chaque sommet et face,
$$
\gathered
w(v) = -2 \qquad \forall v \in V(G) \textrm{,  t.q.  } d_G(v) = 2,\\
w(v) = -1 \qquad \forall v \in V(G) \textrm{,  t.q.  } d_G(v) = 3,\\
w(v) = 0 \qquad \forall  v \in V(G) \textrm{,  t.q.  } d_G(v) = 4
\endgathered
$$
et
$$
\gathered
w(f) = 0 \qquad \forall f \in F(G) \textrm{,  t.q.  } d_G(f) = 4,\\
w(f) = 1 \qquad \forall f \in F(G) \textrm{,  t.q.  } d_G(f) = 5,\\
w(f) \geq 2 \qquad \forall f \in F(G) \textrm{, t.q.  } d_G(f) \geq 6.
\endgathered
$$
\end{frame}

\begin{frame}
Il ne suffit pas de définir les règles de déchargement parmi les sommets, puisqu'il n'y a pas de sommet de charge strictement positive. Nous sommes forcés à utiliser la charge des faces de taille au moins 5 pour éliminer la charge négative des sommets de degré 2 et 3.

Les règles de déchargement:

\begin{enumerate}
\item [(R)] Toute face de taille au moins 5 donne une unité de charge à chaque sommet incident de degré 2 ou 3.
\end{enumerate}
\end{frame}

\begin{frame}
\underbar{Objectives:}

L'objectif est de réduire les configurations suivantes,
\begin{itemize}
\item Un sommet de degré 2 avec 1 pentagone autour de lui.
\item Les sommets de degré 3 avec un pentagone autour d'eux.
\item Une face de grande taille avec beaucoup de sommets de petit degré autour d'elle.
\end{itemize}
\end{frame}

\section{Résultats}
\subsection{Sommet de degré 3}
\begin{frame}
\underbar{3-3-4-3:}
\begin{figure}[ht]
\centerline{
\includegraphics[scale=1]{./Figures/ConfRed.6}
\hfil
\includegraphics[scale=1]{./Figures/ConfRed.106}
}
\caption{La configuration 3-3-4-3 (gauche) et sa réduction (droite).}
\end{figure}
\end{frame}

\begin{frame}
\underbar{Pair de sommets avec 4 carrées:}
\begin{figure}[ht]
\centerline{
\includegraphics[scale=1]{./Figures/ConfRed.4}
\hfil
\includegraphics[scale=1]{./Figures/ConfRed.104}
}
\caption{La configuration 3-3 avec 4 carrés autour de ce pair (gauche) et sa réduction (droite).}
\end{figure}
\end{frame}

\begin{frame}
\underbar{3-3  1 pentagone en haut:}
\begin{figure}[ht]
\centerline{
\includegraphics[scale=1]{./Figures/ConfRed.5}
\hfil
\includegraphics[scale=1]{./Figures/ConfRed.105}
}
\caption{La configuration 3-3 avec un pentagone en haut (gauche) et sa réduction (droite).}
\label{fig:33PH}
\end{figure}
\end{frame}

\begin{frame}
\underbar{3-3 1 pentagone de coté:}
\begin{figure}[ht]
\centerline{
\includegraphics[scale=1]{./Figures/ConfRed.7}
\hfil
\includegraphics[scale=1]{./Figures/ConfRed.107}
}
\caption{La configuration 3-3 avec un pentagone de coté (gauche) et sa réduction (droite).}
\label{fig:33PC}
\end{figure}
\end{frame}

\subsection{Sommet de degré 2}
\begin{frame}
\underbar{2 carrés:}
\begin{figure}[ht]
\centerline{
\includegraphics[scale=1]{./Figures/ConfRed.8}
\hfil
\includegraphics[scale=1]{./Figures/ConfRed.108}
}
\caption{La configuration 2 avec deux carrés (gauche) et sa réduction (droite).}
\label{fig:2CC}
\end{figure}
%Parler des pentagones qu'il faut mettre
\end{frame}

%\begin{frame}
%\underbar{1 carré et 1 pentagone:}
%\end{frame}

\section{Conclusion}
\begin{frame}
\begin{enumerate}
\item Réussi:
\begin{itemize}
\item Développement d'un outil de vérification.
\item 3 nouvelles configurations réductibles parmi plus de 50 essais des paires $H$, $H'$.
\end{itemize}
\item En cours de travail:
\begin{itemize}
\item Réduction pour $v$ de degré 2.
\item Réduction pour $v$ de degré 3.
\item Vérification de connexité et aucune présence des triangles autour des réductions
\end{itemize}
\end{enumerate}

\end{frame}

\begin{frame}
\begin{center}
Merci pour votre attention!
\end{center}
\end{frame}

\end{document}

