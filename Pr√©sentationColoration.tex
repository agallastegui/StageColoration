\documentclass{beamer}
\usepackage{mathrsfs}
\usepackage[fran�ais]{babel}
\usepackage[cp1250]{inputenc}
\usepackage[IL2]{fontenc}
\usepackage{graphicx}
\mode<presentation>
{ \usetheme{Berlin}
%  \setbeamercovered{transparent}
}
\usepackage{times}
\usepackage[T1]{fontenc}

\newtheorem{definition}{D�finition}
\newtheorem{lemme}{Lemme}
\newtheoreme{proposition}{Proposition}
\newtheorem{conjecture}{Conjecture}

\def\ceps{3656}

\title[Coloration des ar�tes des graphes planaires]{Coloration des ar�tes des graphes planaires}
\author[A. Gallastegui and F. Kardo\v s]{\underbar{Antonio Gallastegui} \and Franti\v sek Kardo\v s}
\institute[]{
Laboratoire Bordelais de Recherche Informatique \\
Universit� de Bordeaux\\
}
\date{Bordeaux, 10 juin, 2016}


% If you wish to uncover everything in a step-wise fashion, uncomment
% the following command: 

%\beamerdefaultoverlayspecification{<+->}


\begin{document}

\begin{frame}
  \titlepage
\end{frame}
\begin{frame}
 \tableofcontents
\end{frame}
\section{Introduction}
\subsection{Notions de Base}
\begin{frame}
Un graphe $G$ est un couple $G=(V,E)$ d'ensembles finis, o�
\begin{itemize}
\item $V=V(G)$ est l'ensemble de \emph{sommets} (vertices) de $G$, et
\item $E=E(G)$ est un ensemble de pairs des sommets, appel�es \emph{ar�tes} de $G$.
\end{itemize}
$e$, $e' \in E(G)$ sont adjacents si $e=uv$ et $e'=uv'$ pour $u$, $v$, $v' \in V(G)$, $v \neq v'$.
\pause
Le degr� $d_G(v)$ de $v\in V(G)$ est le nombre de sommets voisins de $v$.
\begin{itemize}
\item $\Delta(G)$ degr� maximum
\item $\delta(G)$ degr� minimum
\end{itemize} 
Figure------
\end{frame}

\begin{frame}
Le graphe $G$ est planaire ssi il peut �tre dessin� sur le plan sans croisement des ar�tes.
Une face $f$, est une r�gion connexe d�limit�e par des ar�tes. Il y a toujours une face infinie.
\pause
\begin{theorem}[Euler, 1750]
Soit $G$ un graphe planaire connexe � $n$ sommets, $m$ ar�tes et $f$ faces. Alors,
$$
n-m+f=2
$$
\end{theorem}
figure---
\end{frame}

\begin{frame}
La \emph{maille} $g(G)$ d'un graphe $G$ est la longueur d'un plus petit cycle dans $G$, s'il en existe un. Sinon $g(G)$ est infinie.

\pause

\begin{proposition}
Tout graphe planaire contient un sommet de degr� au plus $5$.
\end{proposition}




\end{frame}
\subsection{Coloration de Graphes}
\begin{frame}
Soit $G$ un graphe




\end{frame}
\begin{frame}

\end{frame}
\begin{frame}

\end{frame}
\begin{frame}

\end{frame}
\subsection{R�ductibilit� et d�chargement}
\begin{frame}

\end{frame}
\begin{frame}

\end{frame}
\begin{frame}

\end{frame}
\begin{frame}

\end{frame}
\begin{frame}

\end{frame}
\begin{frame}

\end{frame}
\begin{frame}

\end{frame}
\section{R�ductibilit�}
\begin{frame}

\end{frame}
\begin{frame}

\end{frame}
\begin{frame}
\end{frame}
\section{D�chargement}
\section{Description de l'outil}
\section{R�sultats}
\subsection{Pair de sommets de degr� 3}
\subsection{Sommet de degr� 2}
\subsection{Sommet de degr� 3}
\section{Conclusion}
\end{document}

