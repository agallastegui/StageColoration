\documentclass{beamer}
\usepackage{mathrsfs}
\usepackage[utf8]{inputenc}
\usepackage[french]{babel}
\usepackage[T1]{fontenc}
\usepackage{graphicx}
\mode<presentation>
{ \usetheme{Berlin}
%  \setbeamercovered{transparent}
}
\usepackage{times}
\usepackage[T1]{fontenc}

\newtheorem{lemme}{Lemme}
\newtheorem{proposition}{Proposition}

\newtheorem{conjecture}{Conjecture}

\def\ceps{3656}

\title[Coloration des arêtes des graphes planaires]{Coloration des arêtes des graphes planaires}
\author[A. Gallastegui and F. Kardo\v s]{\underbar{Antonio Gallastegui} \and Franti\v sek Kardo\v s}
\institute[]{
Laboratoire Bordelais de Recherche Informatique \\
Université de Bordeaux\\
}
\date{Bordeaux, 10 juin, 2016}


% If you wish to uncover everything in a step-wise fashion, uncomment
% the following command: 

%\beamerdefaultoverlayspecification{<+->}


\begin{document}

\begin{frame}
  \titlepage
\end{frame}
\begin{frame}
 \tableofcontents
\end{frame}
\section{Introduction}
\subsection{Notions de Base}
\begin{frame}
Un graphe $G$ est un couple $G=(V,E)$ d'ensembles finis, où
\begin{itemize}
\item $V=V(G)$ \emph{sommets} de $G$,
\item $E=E(G)$ pairs des sommets, \emph{arêtes} de $G$.
\end{itemize}
$e$, $e' \in E(G)$ sont adjacents si $e=uv$ et $e'=uv'$ pour $u$, $v$, $v' \in V(G)$, $v \neq v'$.

\pause

Le degré $d_G(v)$ de $v\in V(G)$ est le nombre de sommets voisins de $v$.
\begin{itemize}
\item $\Delta(G)$ degré maximum
\item $\delta(G)$ degré minimum
\end{itemize} 
Figure------
\end{frame}

\begin{frame}
Soit $G$ simple. Un \emph{couplage} $M$ est un ensemble d'arêtes deux çà deux non-adjacentes.
---figure couplage maximum, maximal, parfait----


\end{frame}

\begin{frame}
Le graphe $G$ est planaire ssi il peut être dessiné sur le plan sans croisement des arêtes.
Une face $f$, est une région connexe délimitée par des arêtes. Il y a toujours une face infinie.
\pause
\begin{theorem}[Euler, 1750]
Soit $G$ un graphe planaire connexe à $n$ sommets, $m$ arêtes et $f$ faces. Alors,
$$
n-m+f=2
$$
\end{theorem}
figure---
\end{frame}

\begin{frame}
La \emph{maille} $g(G)$ d'un graphe $G$ est la longueur d'un plus petit cycle dans $G$, s'il en existe un. Sinon $g(G)$ est infinie.

\pause

\begin{proposition}
Tout graphe planaire contient un sommet de degré au plus $5$.
\end{proposition}

\pause

Dans un graphe planaire les faces de taille $\geq 3$ peuvent être divisées en deux faces. Appliquant cet opération tant que ce soit possible on n'obtient que des faces triangulaires, c'est une triangulation.

\end{frame}

\subsection{Coloration de Graphes}
\begin{frame}
Soit $G$ graphe, une \emph{coloration propre} de $G$ est application $\varphi: V(G) \to C$ où $C=\{1,2,...,k\}$ ensemble d'entiers, t.q. $\forall uv\in E(G) \Rightarrow \varphi(u) \neq \varphi(v)$. 

Le nombre chromatique $\chi(G)$ est le nombre de couleurs minimum pour colorier $G$

image---coloration nombre chromatique et k-coloration

\end{frame}
\begin{frame}
Soit $G$ un graphe. Une coloration d'arêtes de $G$ est une application $\varphi:E(G) \to C$, où $C=\{1,2,...,k\}$ t.q. $\forall e$, $e' \in E(G)$ adjacentes $\Rightarrow$ $\varphi(e) \neq \varphi(e')$. 

L'indice chromatique $\chi'(G)$ est le nombre de couleurs minimum qu'on nécessite pour colorier les arêtes de $G$.

----image coloration d'aretes et apres indice chromatique etc----
\end{frame}

\begin{frame}
\begin{theorem}[Vizing]
Soit $G$ un graphe planaire. Alors $\Delta(G) \leq \chi'(G) \leq \Delta(G) + 1$.

Si $\chi'(G) = \Delta(G)$ alors $G$ est de classe 1.
\end{theorem}
\pause

\begin{itemize}
\item $\Delta(G) \geq 8$ $\chi'(G)=8$. (Vizing 1965).
\item $\Delta(G) = 7$  $\chi'(G)=7$. (Sanders, Zhao 2001).
\item $\Delta(G) = 5$ ($g(G) \geq 4$) $\chi'(G) = 5$.
\item $\Delta(G) = 4$ ($g(G) \geq 5$) $\chi'(G) = 4$.
\item $\Delta(G) = 4$ ($g(G) \geq 4$) $\chi'(G) = 4$?
\end{itemize}


\end{frame}

\section{Réductibilité et déchargement}
\begin{frame}
Soit $G$ graphe avec arêtes coloriées et $G[c1,c2]$ sous graphe de $G$ induit par arêtes coloriées $c1$ et $c2$. 

Tout composante connexe $C$ de $G[c1,c2]$ est une chaîne à laquelle renversants les couleurs on obtient nouvelle coloration de $G$.

Pour $e\in E(G)$ t.q. $\varphi(e)=c$ et une autre couleur $c'$. Alors il existe une et une seule chaîne de $G[c,c']$ contenant $e$. C'est \emph{$(c, c')$-chaîne de Kempe}. 
\end{frame}

\subsection{$\Delta(G)=5$, $g(G)\geq 4$, $\chi'(G)=5$}
\begin{frame}
\begin{proposition}
Soit $G$ un graphe planaire avec $\Delta(G) = 5$ et $g(G) \geq 4$. Alors $\chi'(G)=5$.
\end{proposition}
Preuve.
\end{frame}
\begin{frame}
Propriétés Structurelles:
\begin{lemme}
Il n'existe pas de sommet $v$ de $G$ tel que $d_G(v) = 1$.
\label{le:1}
\end{lemme}
\end{frame}

\begin{frame}
\begin{lemme}
Soit $v \in V(G)$ tel que $d_G(v) = 2$. Alors $\forall u \in N(v)$, $d_G(u) = 5$.
\label{le:2}
\end{lemme}
\end{frame}

\begin{frame}
\begin{lemme}
Soit $v\in V(G)$ tel que $d_G(v) = 3$. Alors, $\forall u \in N(v)$, $4\le d_G(u) \le 5$.
\label{le:33}
\end{lemme}
\end{frame}

\begin{frame}
\begin{lemme}
Soit $v\in V(G)$ tel que $d_G(v) = 3$. Alors, au moins deux voisins de $v$ sont de degré $5$.
\label{le:434}
\end{lemme}
\end{frame}

\begin{frame}
\begin{lemme}
Soit $uv \in E(G)$ tel que $d_G(u) = 2$ et $d_G(v) = 5$. Alors, tous les voisins de $v$ sauf $u$ sont de degré $5$. 
\label{le:254}
\end{lemme}
\end{frame}

\begin{frame}
\begin{lemme}
Soit $v \in V(G)$ tel que $d_G(v) = 5$. Alors, $v$ a au plus deux voisins de degré $3$.
\label{le:5333}
\end{lemme}
\end{frame}

\begin{frame}
Déchargement:

La charge initiale est une application  $w: V(G) \to \mathbb{R}$  la fonction définie par
 $$
 \gathered
 w(v) = d_G(v) - 4 \qquad \textrm{pour tout $v\in V(G)$,} \\
 w(f) = d_G(f) - 4 \qquad \textrm{pour tout $f\in F(G)$,} 
\endgathered 
$$

$$ 
\gathered
w(v) = -2 \qquad \textrm{pour tout $v \in V(G)$, tel que } d_G(v) = 2, \\
w(v) = -1 \qquad \textrm{pour tout $v \in V(G)$, tel que } d_G(v) = 3, \\
w(v) =  0 \qquad \textrm{pour tout $v \in V(G)$, tel que } d_G(v) = 4, \\
w(v) =  1 \qquad \textrm{pour tout $v \in V(G)$, tel que } d_G(v) = 5.
\endgathered
$$

La charge initiale de toutes les faces est positive puisque $g(G)\ge 4$.
\end{frame}

\begin{frame}
\begin{lemme}
Soit $G$ un graphe planaire connexe. Alors,
$$ \sum_{v \in V(G)} (d_G(v) - 4) + \sum_{f\in F} (d_G(f) -4) = -8.$$
\label{le:charge}
\end{lemme}
Preuve.
$$
\gathered 
\sum_{v \in V(G)} (d_G(v) - 4) + \sum_{f\in F} (d_G(f) -4) = 2m -4n + 2m - 4f = \\
 =-4(n - m + f) = -4\cdot 2 = -8.
\endgathered
$$
\end{frame}

\begin{frame}
Les règles de déchargement:
\begin{enumerate}
\item[(R1)] Tout sommet de degré 5 donne une unité de charge à chacun de ses voisins de degré 2.
\item[(R2)] Tout sommet de degré 5 donne $1/2$ de charge à chaque voisin de degré 3.
\end{enumerate}
\end{frame}

\subsection{$\Delta(G)=4$, $g(G)\geq 5$, $\chi'(G)=4$}
\begin{frame}
\begin{proposition}
Soit $G$ un graphe planaire avec $\Delta(G) = 4$ et $g(G) \geq 5$. Alors $G$ est $4$-arête-coloriable.
\end{proposition}
Preuve.
\end{frame}

\begin{frame}
Propriétés Structurelles:
\begin{lemme}
Il n'existe pas de sommet $v$ de $G$ tel que $d_G(v) = 1$.
\label{le:14}
\end{lemme}
\end{frame}

\begin{frame}
\begin{lemme}
Soit $v \in V(G)$ tel que $d_G(v) = 2$. Alors $\forall u \in N(v)$, $d_G(u) = 4$.
\label{le:23}
\end{lemme} 
\end{frame}

\begin{frame}
\begin{lemme}
Soit $uv \in E(G)$ tel que $d_G(u) = 2$ et $d_G(v) = 4$. Alors, tous les voisins de $v$ sauf $u$ sont de degré $4$.
\label{le:243}
\end{lemme}
\end{frame}

\begin{frame}
\begin{lemme}
Soit $v \in V(G)$ tel que $d_G(v)=3$. Alors, $v$ a au moins deux voisins de degré $4$.
\label{le:333}
\end{lemme}
\end{frame}

\begin{frame}
Déchargement:

La charge initiale est une application  $w: V(G) \to \mathbb{R}$  la fonction définie par

$$
\gathered
w(v) = \frac{3}{2} d_G(v) - 5 \qquad \textrm{pour tout $v\in V(G)$,} \\
w(f) = d_G(f) - 5 \qquad \textrm{pour tout $f\in F(G)$.}
\endgathered
$$

\pause

$$
\gathered
w(v) = -2 \qquad \textrm{pour tout $v \in V(G)$, tel que } d_G(v) = 2,\\
w(v) = -\frac{1}{2} \qquad \textrm{pour tout $v \in V(G)$, tel que } d_G(v) = 3,\\
w(v) =  1 \qquad \textrm{pour tout $v \in V(G)$, tel que } d_G(v) = 4.
\endgathered
$$

D'ailleurs, la charge initiale de toutes les faces est positive puisque $g(G) \geq 5$.
\end{frame}

\begin{frame}
\begin{lemme}
Soit $G$ un graphe planaire connexe. Alors,
$$
\sum_{v \in V(G)} \left(\frac{3}{2} d_G(v) - 5\right) + \sum_{f\in F(G)} \left(d_G(f) -5\right) = -10.
$$
\label{le:cha2}
\end{lemme}
Preuve.
$$ 
\gathered
\sum_{v \in V(G)} \left(\frac{3}{2}d_G(v) - 5\right) + \sum_{f\in F(G)} \left(d_G(f) -5\right) =\\
= 3m -5n + 2m - 5f = -5(n - m + f) = -5 \cdot 2 = -10.
\endgathered
$$
\end{frame}

\begin{frame}
Les règles de déchargement:
\begin{enumerate}
\item[(R1)] tout sommet de degré 4 donne $1$ de charge à chacun de ses voisins de degré 2.
\item[(R2)] tout sommet de degré 4 donne $1/4$ de charge à chaque voisin de degré 3.
\end{enumerate}
Dans ce cas là aussi, la somme totale de la charge reste invariante et la charge de toutes les faces reste positive.
\end{frame}


\section{Réductibilité}
\begin{frame}
Les démonstrations de Lemmes \ref{le:1}-\ref{le:5333} et \ref{le:14}-\ref{le:333} ont toutes la même structure: La supposition que le contre-exemple minimal $G$ contienne un arrangement de sommets de certains degrés entraîne une contradiction avec non-colorabilité de $G$.
  

Nous allons généraliser cette notion.
\end{frame}

\begin{frame}
Soit $H$ un graphe étiqueté avec une fonction $\gamma$: $V(G) \to \mathbb{N}$ telle que $\forall v \in V(H)$ on a $\gamma(v) \geq d_H(v)$. Le graphe $H$ est une \emph{configuration} dans $G$ s'il existe un sous-graphe $X$ de $G$ et un isomorphisme $\xi$: $H \to X$ tel que $\forall v \in V(H)$, $\gamma(v) = d_G(\xi(v))$. 

\pause

Dans la suite, lorsque nous parlons d'une configuration $H$ dans un graphe $G$, nous identifions $H$ et sa copie $X=\xi(H)$ dans $G$.
\end{frame}

\begin{frame}
Soit $H$ une configuration de $G$. Notons
$$
\partial_G(H) = \{e =uv \in E(G) : u \in V(H)\textrm{ et }v \in V(G \setminus H)\},
$$ 
l'ensemble des arêtes sortantes de $H$ et $r=|\partial_G(H)|$.
\pause

Notons $H^* = H \cup \partial_G(H)$ et $G'=G\setminus H \cup \partial_G(H)$.
\end{frame}

\begin{frame}
Si $\varphi:E(G)\to \{1,2,3,4\}$ est une 4-arête-coloration de $G$, alors $\varphi$ induit une coloration de $\partial_G(H)$, de $H^*$, et de $G'$. Pour capturer la structure de la coloration $\varphi$ dans $G'$, nous introduisons les notions et notations suivantes :
\end{frame}

\begin{frame}
Soit 
\begin{itemize}
\item $G$ un graphe avec $\Delta(G)=4$ et $H$ une configuration de $G$.
\item $\varphi:E(G')\to \{1,2,3,4\}$ une 4-arête-coloration de $G'$. 
\item $c, c' \in \{1,2,3,4\}$, $c\ne c'$ deux couleurs quelconques. 
\end{itemize}

Une coloration $\varphi':E(G)\setminus E(H)\to \{1,2,3,4\}$ est \emph{adjacente} à $\varphi$, si $\varphi'$ peut être obtenu à partir de $\varphi$ en alternant les couleurs d'une $(c,c')$-chaîne de Kempe de $e$ pour un choix d'une arête $e\in\partial_G(H)$ et des couleurs $c,c'\in\{1,2,3,4\}$. 

Notons $N(\varphi)$ l'ensemble des colorations adjacentes à $\varphi$.
\end{frame}

\begin{frame}
Soit 
\begin{itemize}
\item $G$ un graphe avec $\Delta(G)=4$ et $H$ une configuration de $G$.
\item $\varphi:E(G')\to \{1,2,3,4\}$ une 4-arête-coloration de $G'$. 
\item $c, c' \in \{1,2,3,4\}$, $c\ne c'$ deux couleurs quelconques. 
\end{itemize}

\pause


Le \emph{graphe résiduel}
$\mathcal{X}^{\varphi}_{c,c'}$ est défini par
\begin{itemize}
\item $V(\mathcal{X}^{\varphi}_{c,c'}) = \partial_G(H)$, 
\item $E(\mathcal{X}^{\varphi}_{c,c'}) = \{ e_ie_j$ tels que $(c,c')$-chaîne de Kempe de $e_i$ est la même que la $(c,c')$-chaîne de Kempe de $e_j\}$.
\end{itemize}

\pause

Le graphe résiduel est donc un couplage planaire de $\partial_G(H)$.
\end{frame}

\begin{frame}
\begin{proposition}
Soit 
\begin{itemize}
\item $G$ avec $\Delta(G)=4$ et $H$  de $G$.
\item $\varphi:E(G')\to \{1,2,3,4\}$ 4-arête-coloration de $G'$. 
\item $\varphi' \in N(\varphi)$ une coloration adjacente à $\varphi$
\item $\beta = \varphi|_{\partial_G(H)}$ et $\beta' = \varphi'|_{\partial_G(H)}$
\end{itemize}
Alors,
$$
|\{e\in \partial_H(G) : \beta(e)\ne \beta'(e)\}| \in \{1,2\},
$$
de plus, on a \\
$|\{e\in \partial_H(G) : \beta(e)\ne \beta'(e)\}|=1$ $\Leftrightarrow$ $d_{\mathcal{X}_{\varphi}^{c,c'}}(e)=0$, et \\
$|\{e\in \partial_H(G) : \beta(e)\ne \beta'(e)\}|=1$ $\Leftrightarrow$ $d_{\mathcal{X}_{\varphi}^{c,c'}}(e)=1$. 
\end{proposition}
\end{frame}

\begin{frame}
Considérons des coloration de $H$ 
\end{frame}

\begin{frame}
Soit $H$ de $G$ et $\beta$ une coloration de $\partial_G(H)$. 

Alors, $\beta$ \emph{s'étend} vers $H$ si $\exists \psi$ de $H*$ t.q. $\psi|_{\partial_G(H)}$

$$
\Psi_0 = \{\beta : \exists \psi : E(H^*)\to\{1,2,3,4\} \textrm{t.q.} \psi|_{\partial_G(H)} = \beta\}.
$$
Pour $i\in\mathbb{N}$, soit
$$
\gathered
\Psi_{i+1} = 
\{ 
\beta : 
\beta\in \Psi_i
\textrm{ ou }
\exists c,c'\in \{1,2,3,4\},\, 
\forall X \quad
N^X_{c,c'}(\beta) \cap \Psi_i \neq \emptyset 
\},\\
\Psi(H)=\bigcup_{i\in \mathbb{N}} \Psi_i(H).
\endgathered
$$
\end{frame}

\begin{frame}
\begin{theorem}
Soit 
\begin{itemize}
\item $H$  de $G$. 
\item $H'$ une autre configuration t.q. $\partial(H)=\partial(H')$.
\item $G'$ le graphe obtenu à partir de $G$ en remplaçant $H$ par $H'$.
\end{itemize}

Si $\Psi_0(H')\subseteq \Psi(H)$, alors, pour toute 4-arête-coloration $\varphi'$ de $G'$ il existe une 4-arête-coloration $\varphi$ de $G$.
\label{th:red}
\end{theorem}

Preuve.
il suffit de modifier la coloration $\varphi'$ le long des chaînes de Kempe un nombre fini de fois afin d'obtenir une coloration qui s'étend directement vers $H$.
\end{frame}

\begin{frame}
Théorème \ref{th:red} permet d'automatiser la vérification de réductibilité de configurations -- on n'a pas besoin de connaître la structure du graphe $G$, il suffit de deviner la réduction (la configuration $H'$) et vérifier que la condition de la définition dé réductibilité est satisfaite.
\end{frame}

\section{Déchargement}
\begin{frame}
Soit $G$ un graphe planaire. V(G) ensemble de sommets et F ensemble de faces de $G$.

La \emph{charge initiale} des sommets et des faces

$$
\gathered
w: V(G) \to \mathcal{R} \qquad \textrm{telle que}\\
w(v) = d_G(v) - 4 \qquad \textrm{pour tout $v \in V(G)$}\\
\endgathered
$$

et également

$$
\gathered
w: F \to \mathcal{R} \qquad \textrm{telle que}\\
w(f) = d_G(f) - 4 \qquad \textrm{pour tout $f \in F$}\\
\endgathered
$$
\end{frame}

\begin{frame}
\begin{lemme}
Soit $G$ un graphe planaire et $w$ la application de la charge initiale. Alors la charge initiale totale du graphe est $-8$ (lemme \ref{le:charge}).
\end{lemme}

Preuve.
$$
\gathered
\sum_{u \in V(G)} (d_G(v) - 4) + \sum_{f \in F} (d_G(f) - 4) = 2m - 4n + 2m - 4f =\\
= -4( n - m + f) = -4 \cdot 2 = -8.
\endgathered
$$
\end{frame}

\begin{frame}
Alors les charges pour chaque sommet et face,
$$
\gathered
w(v) = -2 \qquad \forall v \in V(G) \textrm{,  t.q.  } d_G(v) = 2\\
w(v) = -1 \qquad \forall v \in V(G) \textrm{,  t.q.  } d_G(v) = 3\\
w(v) = 0 \qquad \forall  v \in V(G) \textrm{,  t.q.  } d_G(v) = 4
\endgathered
$$
Et également pour les faces on obtient,
$$
\gathered
w(f) = 0 \qquad \forall f \in F(G) \textrm{,  t.q.  } d_G(f) = 4\\
w(f) = 1 \qquad \forall f \in F(G) \textrm{,  t.q.  } d_G(f) = 5\\
w(f) \geq 2 \qquad \forall f \in F(G) \textrm{, t.q.  } d_G(f) \geq 6
\endgathered
$$
\end{frame}

\begin{frame}
Les règles de déchargement:

\begin{enumerate}
\item [(R1)] $\forall v \in V(G)$ tel que $d_G(v) = 2$, $v$ donne une unité de charge négative $(-1)$ à chaque des deux faces de taille 5 incidentes à $v$ ou donne toute sa charge négative $(-2)$ à une face incidente $f$ telle que $d_G(f) \geq 6$.
\item [(R2)] $\forall v \in V(G)$ tel que $d_G(v) = 3$, $v$ donne une unité de charge négative $(-1)$ à une face voisine $f$ telle que $d_G(f) \geq 5$.
\item [(R3)] $\forall v \in V(G)$ tel que $d_G(v) = 4$, $v$ ne distribue aucune charge car il ne l'a pas.
\end{enumerate}
\end{frame}


\section{Description de l'outil}
\subsection{Propriétés Structurelles}
\begin{frame}
Les propriétés des Lemmes \ref{le:14}-\ref{le:243} se donnent aussi dans ce cas-là.
\end{frame}

\begin{frame}
\begin{lemme}
Soit $G$ un contre-exemple minimal et $u$ et $v$ deux sommets voisins tels que $d_G(u)=d_G(v) = 3$. Alors parmi les 4 faces autour de ce paire de sommets il y a au moins deux pentagones, et tous les sommets autour des faces sauf $v$ et $u$ ont degré 4.
\label{le:33pent}
\end{lemme}
\end{frame}

\begin{frame}
\begin{lemme}
Pour un pair de sommets $u$ et $v$ de degré $3$ adjacentes, la distance à un autre sommet de degré $3$ est au moins $2$.
\label{de:3343}
\end{lemme}
\end{frame}

\begin{frame}
\begin{lemme}
Soit $G$ un contre-exemple et $v$ un sommet de $G$ tel que $d_G(v) = 2$. Alors, il existe autour de $v$ deux pentagones ou un hexagone et tous les sommets autour des faces sont de degré 4.
\label{le:2pent}
\end{lemme}
\end{frame}

\begin{frame}
\begin{lemme}
Soit $G$ un contre-exemple minimal et $v \in V(G)$ tel que $d_G(v) = 3$.
\label{le:3}
\end{lemme}
\end{frame}

\begin{frame}
Ces deux derniers cas n'ont pas encore été prouvés et nous n'avons pas trouvé leurs réductions non plus.
\end{frame}

\section{Résultats}
\begin{frame}

\end{frame}
\subsection{Pair de sommets de degré 3}
\begin{frame}
4 Carrés:
\end{frame}

\begin{frame}
3 Carrés et 1 Pentagone:
\end{frame}

\begin{frame}
3-3-4-3:

\end{frame}

\subsection{Sommet de degré 2}
\begin{frame}
2 Carrés
\end{frame}

\begin{frame}
1 Carré et 1 Pentagone
\end{frame}

\subsection{Sommet de degré 3}
\begin{frame}
3 Carrés
\end{frame}

\section{Conclusion}
\begin{frame}
\begin{enumerate}
\item Réussi
\begin{itemize}
\item Développement d'un outil de vérification.
\item 3 nouvelles configurations réductibles parmi plus de 50 essais.
\item Des propriétés qui soutiennent les règles de déchargement.
\end{itemize}
\item Failli
\begin{itemize}
\item Réduction pour $v$ de degré 2 avec 1 carré et 1 pentagone.
\item Réduction pour $v$ de degré 3 avec 3 carrés
\item Vérification de connexité et aucune présence des triangles autour des réductions
\end{itemize}
\end{enumerate}

\end{frame}

\begin{frame}
\begin{center}
Merci pour votre attention!
\end{center}
\end{frame}

\end{document}

