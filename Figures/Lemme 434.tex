\documentclass[10pt,a4paper]{article}
\usepackage[utf8]{inputenc}
\usepackage[french]{babel}
\usepackage[T1]{fontenc}
\usepackage{amsmath}
\usepackage{amsfonts}
\usepackage{amssymb}
\usepackage{makeidx}
\usepackage{graphicx}
\usepackage{color}
\usepackage[left=2cm,right=2cm,top=2cm,bottom=2cm]{geometry}
\newtheorem{definition}{Définition}
\newtheorem{theorem}{Théorème}
\newtheorem{proposition}{Proposition}
\newtheorem{exemple}{Exemple}
\newtheorem{lemme}{Lemme}


\newcommand{\ep}{{\hfill $\square$}}

\addtolength{\parskip}{5pt}

\begin{document}

Soit $G$ un graphe avec les arêtes coloriées. 
Considérons le sous-graphe $H^{c_1,c_2}$ de $G$ induit par les arêtes coloriés $c_1$ et $c_2$. Par la définition d'une coloration, aucun sommet de $H^{c_1,c_2}$ ne peut avoir un degré supérieur à 2. Par conséquent, toute composante connexe de $H^{c_1,c_2}$ est une chaîne ou un cycle. Si on choisit une composante connexe $C$ de $H^{c_1,c_2}$, qu'on échange les couleurs $c_1$ et $c_2$ le long de $C$ et qu'on garde la couleur de toutes les autres arêtes, on obtiendra une autre coloration de $G$.

Étant donné une arête $e$ de $G$ de couleur $c$ et une autre couleur $c'$, il existe une et une seule composante connexe de $H^{c,c'}$ contenant $e$.
Nous allons appeler celle-là la \emph{$(c,c_1)$-chaîne de Kempe} de $e$. 




\begin{lemme}
Soit $v$ un sommet de $G$ tel que $d(v) = 3$. Alors, pour tout voisin $u$ de $v$ on a $4\le d(u) \le 5$.
\end{lemme}

Preuve. De la même manière que la démonstration du lemme précédent, si $v$ avait un voisin $u$ de degré (au plus) 3, l'arête $uv$ serait adjacente à (au plus) quatre autres arêtes, alors, n'importe quelle coloration de $G\setminus uv$ donnerait une coloration de $G$, l'absurde.

\begin{lemme}
Soit $v$ un sommet de $G$ tel que $d(v) = 3$. Alors, il existe au moins deux sommets $u_1$, $u_2 \in N(v)$ tell que $d(u_1) = d(u_2)= 5$.
\end{lemme}

Preuve.
Supposons, par l'absurde, qu'il n'existe que $u_1 \in N(v)$ tel que $4\le d(u_1) \le 5$ et pour $ u_2$, $u_3 \in N(v)$ on a $d(u_2) = d(u_3) = 4$. 

Soit $e_i=vu_i$, $i=1,2,3$ ; soit $\{e_i,e_{i1},e_{i2},e_{i3}\}$ l'ensemble des arêtes incidentes à $u_i$, $i=2,3$. 
Considérons une coloration $\varphi$ du sous-graphe $G' = G \setminus e_3$. Afin que celle-ci ne s'étende pas en une coloration de $G$, on peut supposer, sans perdre de la généralité, que $\varphi(e_1) = 1$, $\varphi(e_2) = 2$, $\varphi(e_{31}) = 3$, $\varphi(e_{32}) = 4$ et $\varphi(e_{33}) = 5$.

Si les $(2,3)$-chaînes de Kempe des arêtes $e_2$ et $e_{31}$ sont distinctes, alors en alternant les couleurs d'une parmi elles on obtiendrait une coloration de $G'$ dans laquelle les arêtes adjacentes à $e_3$ ne seraient pas coloriées de cinq couleurs différentes, ce qui est absurde.
Donc, il existe une $(2,3)$-chaîne de Kempe reliant $e_2$ et $e_{31}$ ; il y a ainsi une arête incident à $u_2$ coloriée 3, disons que $\varphi(e_{21})=3$. Voir Figure \ref{fig:434} pour illustration.

Similairement, il existe une $(2,4)$-chaîne (une $(2,5)$-chaîne) reliant $e_2$ et $e_{32}$ ($e_2$ et $e_{33}$, respectivement). Par ailleurs, il n'y a pas d'arête coloriée 1 incidente à $u_2$.

Considérons maintenant les $(1,3)$-chaînes de Kempe. Par le même argument, la chaîne qui commence par $e_1$ se termine forcément par $e_{31}$. Donc, on est libre à alterner la chaîne de Kempe de $e_{21}$, recolorier $\varphi(e_2)=3$ et poser  
$\varphi(e_3)=2$, créant ainsi une coloration de $G$, une contradiction. 
\ep 
    
\begin{figure}[ht]
\centerline{
\begin{tabular}{ccc}
\begin{tabular}{c}
\includegraphics[scale=1]{434.1}
\end{tabular}
&
$\longrightarrow$
&
\begin{tabular}{c}
\includegraphics[scale=1]{434.2}
\end{tabular}
\end{tabular}
}
\caption{Un sommet de degré trois avec deux voisins de degré quatre constitue une configuration réductible. }
\label{fig:434}
\end{figure}


\end{document}

